Grâce aux co-marquage de \gls{musk} et \gls{gfap}, j'ai pu montré que dans le cerveau, \gls{musk} est exprimé par les astrocytes. Cependant,  un doute reste sur la spécificité du marquage de l'anticorps anti-\gls{musk}. Hessert \emph{et al.} en 2006 ont généré une lignée de souris \gls{musk}/LoxP croisée avec une souris où le gène de la recombinase Cre est sous promoteur de la créatine kinase \cite{Hesser2006}. Dans cette souris Cre/LoxP, \gls{musk} est inactivé dans les cellules musculaires au cours du développement, à la différence des souris KO. On peut alors imaginer croiser la lignée \gls{musk}/LoxP avec une lignée de souris Cre sous contrôle d'un promoteur nerveux (exprimé à la fois par les neurones et par les cellules glial) pour obtenir un modèle de souris \gls{musk} KO spécifique du cerveau. Cette lignée permettrait d'étudier l'effet du récépteur dans le cerveau, et permettrait également de confirmer la spécificité du marquage.

Un autre moyen de confirmer le marquage de \gls{musk} serait de réaliser un co-marquage avec un autre anticorps dirigé contre le récepteur. J'ai appris récemment l'existence d'un tel anticorps utilisé par l'équipe de Lin Mei, utilisable en immunofluorescence. Le co-marquage de deux anticorps dirigé contre deux régions différentes du récepteur serait un bon indice sur la spécificité du marquage. 

Pour l'expression de \gls{musk}, de nombreux gènes sont exprimés de manière différentes entre les deux hémisphères, notamment entre les deux hippocampes \cite{Moskal2006}. Cette asymétrie Gauche-Droite est un processus essentiel notamment à la formation de la mémoire \cite{Shimbo2018}. \gls{musk} suit asymétrie et apparaît plus fortement exprimé dans l'hippocampe gauche que dans l'hippocampe droit. Cependant, cette tendance ne ressort par chez les souris mutante. Mais le gène chez ces dernière s'est amplifié tard (après le 35\up{ème} cycle). Ces résultats ne sont donc pas en eux-mêmes interprétable. Il ne serait pas inenvisageable que ce qui ai été amplifié soit du bruit de fond. Cependant, on peut voir que la différence avec les souris sauvages est notable. On peut donc poser deux hypothèse : Soit effectivement, l'expression de \gls{musk} est fortement réduite chez les mutants, soit la mutation empêche l'appariement des primers sur le gène. Ces primers fournis par Qiagen\texttrademark sont propriétaire et leur séquence exacte n'est pas connue. Cependant, la région où les primers s'accrochent est éloignée de 1000pbs de la région du \gls{crd}, donc cette dernière hypothèse est peu probable.

Il serait interessant de voir l'évolution de l'expression de \gls{musk} au cours du développement : on pourrait alors faire des \gls{qpcr} à différents stades embryonnaire et néo-natal.

Comparaison Immuno/qPCR
Cultures
Hypothèse rôle Musk
Tempéré résultats neun : taille effet ? n suffisant ? variations lors coupes/mesures.
 A finir