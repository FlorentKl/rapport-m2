%Structure du Cerveau 
J'ai pu montrer durant mon stage que bien que l'organisation globale du cerveau des souris \mcrd soit normale, la structure de l'hippocampe est modifiée, avec le rétrécissement de la région CA3 dans la partie caudale de l'hippocampe et une modification de la taille de la couche granulaire du Gyrus Denté (diminution de la zone infrapyramidale dans la partie rostrale, augmentation de la zone infrapyramidale et diminution de la zone suprapyramidale dans la partie caudale). Ces résultats restent à confirmer en augmentant le nombre d'individus testés (3 souris sauvages et 4 souris mutantes pour l'instant).

%Marquage, loca musk
J'ai également montré que le récepteur \gls{musk} était exprimé à la fois par les astrocytes et par les neurones, grâce à de l'immunomarquage sur des coupes de cerveaux et sur des cultures cellulaires d'hippocampe. Dans les coupes de cerveaux, le marquage observé de \gls{musk} colocalise avec un marquage de \gls{gfap}, marqueur des astrocytes. D'après leur morphologie, les astrocytes visualisés sont de type fibreux. Le même type de marquage est observé dans la moelle épinière. Mes marquages révèlent que le récepteur est présent dans des régions discrètes du cerveau (Hippocampe, Habenula, Fasciculus retroflexus, Pédoncule cérébraux notamment), qui sont des régions riches en fibres nerveuses, à la différence de régions plus denses en neurones (Cortex, Substance grise de la moelle épinière...) où ce marquage astrocytaire est beaucoup moins présent. La forme des astrocytes marqués par \gls{musk} est similaire entre les coupes de tissus et les cultures cellulaires.

Ces résultats tranchent avec ce qui avait été décrit précédemment grâce à une \acrlong{his} de \gls{musk}, où les auteurs indiquaient que presque toutes les zones du cerveaux sont marquées \cite{Garcia-Osta2006}. Ces auteurs se sont également concentrés sur l'étude des neurones, et n'ont pas décrit la présence de marquage dans des cellules gliales, mais d'autres articles ont cependant montré que \gls{musk} était également exprimé par les astrocytes \cite{Sun2016}.

Le co-marquage de cultures cellulaires d'hippocampe avec \gls{musk} et \gls{map2} a permis de montrer que le récepteur était également exprimé par les neurones dans les dendrites et le soma. Ce marquage est donc cohérent avec l'observation par \acrlong{his} de la présence de \gls{musk} sur des coupes de cerveau ou de cultures primaires de neurones d'hippocampes \cite{Garcia-Osta2006}. Cependant, on ne retrouve pas de marquage neuronale sur les coupes de cerveaux. Cela peut être dû à une très faible intensité du marquage qui se retrouverait noyé dans le bruit de fond. En effet, le marquage observé dans les dendrites sur les cultures est moins intense que celui observé dans les astrocytes. Le niveau d'expression par les neurones de \gls{musk} pourrait donc être moins important que dans les astrocytes, ce qui avait déjà été décrit par \gls{qpcr} dans des cultures issues de jeunes souriceaux (stade E18 ou P2-P3 en fonction du type cellulaire étudié) \cite{Sun2016}. 

On peut également observer sur les cultures cellulaires un marquage somatique des neurones (co-localisation avec \gls{map2}). Au niveau des coupes, on observe un léger marquage dans le soma, que j'ai tout d'abord considéré comme du bruit de fond, mais qui serait donc spécifique. Cependant, au vu de la qualité du marquage et des coupes contrôles (sans anticorps secondaires) où de telles structures semblent également être présentes, il n'est pas possible de trancher entre ces deux hypothèses.

Pour le marquage des dendrites, plusieurs hypothèses peuvent rendre compte des différences de marquages entres coupes et cultures. Par exemple, le stade de différenciation des cellules pourrait ne pas être le même \emph{in vivo} et \emph{in vitro}. Les coupes de cerveaux sont issues d'animaux âgés de 30 jours, tandis que les cultures cellulaires sont issues d'embryon au stade E16 et cultivées 14 jours. Cependant, en observant les coupes d’embryon sauvages à E18.5, on ne voit pas non plus l’apparition d’un marquage MuSK dans les dendrites neuronales. Une autre hypothèse pourrait être la représentativité anormale de certains types de cellules dans les cultures cellulaires. Ces cultures présentent sont composées principalement de neurones (90\% de neurones pyramidaux et 10\% d'interneurones) et de quelques cellules gliales seulement (majoritairement des astrocytes). Il peut y avoir également présence de quelques cellules venant des méninges. Les cultures sont donc loin des proportions observables dans le cerveaux, où l'ont considère qu'il y a autant de cellules gliales que de neurones.Les cultures cellulaires ayant été faites afin de favoriser la présence de neurones, cela pourrait modifier le profil d'expression de \gls{musk}. Dans ce contexte, on pourrait imaginer une régulation de \gls{musk} dans les neurones par d'autres types cellulaires. Enfin, le marquage du soma pourrait être également dû à un début de dégénérescence des cellules lié aux conditions de cultures.

Les astrocytes sont un élément crucial de la synapse, que l'on qualifie désormais de "tri-partite" \cite{Araque1999, Perea2009}. On pourrait penser que \gls{musk} va jouer le même rôle qu'à la \gls{jnm}, et permettre la mise en place de l'élément post-synaptique (ici le prolongement de l'astrocyte et le dendrite neuronal). Cependant, on observerait alors un marquage ponctiforme correspondant aux synapses, et non un marquage continu dans les prolongements. Comme \gls{musk} semble être présent dans toute la cellule astrocytaire, et dans les dendrites et soma des neurones, on peut alors imaginer différents rôles du récepteur. Ce dernier pourrait être impliqué dans la neuroprotection ou la prolifération, notamment au travers de la signalisation \Gls{wnt} \cite{Toledo2008, Cerpa2009}, ou bien encore comme senseur extracellulaire.

Concernant l'expression de \gls{musk}, de nombreux gènes sont exprimés de manières différentes entre les deux hémisphères, notamment entre les deux hippocampes \cite{Moskal2006}. Cette asymétrie Gauche-Droite est un processus essentiel notamment à la formation de la mémoire \cite{Shimbo2018}. \gls{musk} est donc un nouvel exemple d'expression asymétrique, en étant plus fortement exprimé dans l'hippocampe gauche de la souris que dans l'hippocampe droit. On peut supposer que l'expression de \gls{musk} est semblable entre les individus sauvages et mutants, car l'immunomarquage du récepteur est semblable entre les deux types d'individus.

Ce travail est l'un des premiers à fournir un aperçu de la localisation de la protéine \gls{musk} dans le cerveau. Il a permis de montrer que le récepteur \gls{musk} était localisé dans des régions discrètes du cerveau, et par immunomarquage de confirmer sa présence dans des neurones et astrocytes, qui n'avait alors été observée que par \gls{his} et \gls{wb} chez l'adulte, et par \gls{qpcr} dans des cultures de cellules neurales embryonnaires. J'ai également pu montré que \gls{musk} pourrait jouer un rôle dans la distribution des neurones dans certaines régions de l'hippocampe, au travers notamment de son domaine \gls{crd}. J'ai enfin montré que l'expression du récepteur \gls{musk} était asymétrique entre les deux hippocampes. De nombreux travaux dans le domaine restent cependant nécessaires.