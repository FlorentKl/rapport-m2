%Structure du Cerveau 
J'ai pu montrer durant mon stage que bien que l'organisation globale du cerveau des souris \mcrd restait inchangée, la structure de l'hippocampe allait être modifié, avec le rétrécissement de la région CA3 dans la partie caudale du cerveau, ou bien une modification de la taille de la couche granulaire du Gyrus Denté. Cependant, il est possible que j'ai été biaisé lors de mes mesures ou du choix des coupes. Il serait alors intéressant de refaire ces mesures en augmentant le nombre d'individu, qui était de 3 souris souris sauvages et 4 souris mutantes.

%Marquage, loca musk
Grâce aux co-marquage de \gls{musk} et \gls{gfap}, j'ai pu montré que dans le cerveau, \gls{musk} est exprimé dans les astrocytes qui semblent être de type fibreux, car ils possèdent de long prolongements lisses. Cependant, un doute persiste sur la spécificité du marquage de l'anticorps anti-\gls{musk}. Hessert \emph{et al.} en 2006 ont généré une lignée de souris \gls{musk}/LoxP croisée avec une souris où le gène de la recombinase Cre est sous promoteur de la créatine kinase \cite{Hesser2006}. Dans cette souris Cre/LoxP, \gls{musk} est inactivé dans les cellules musculaires après la naissance, à la différence des souris KO, ou le gène est absent dès le début du développement. On peut alors imaginer croiser la lignée \gls{musk}/LoxP avec une lignée de souris Cre sous contrôle d'un promoteur nerveux (exprimé à la fois par les neurones et par les cellules glial) pour obtenir un modèle de souris \gls{musk} KO spécifique du cerveau. Cette lignée permettrait d'étudier l'effet de la délétion de \gls{musk} sur le comportement, et permettrait également de confirmer la spécificité du marquage astrocytaire.

Il apparaît que les structures marquées par \gls{musk} sont des régions riches en fibres nerveuses et en astrocytes. Les régions denses en corps cellulaire (substance grise de la moëlle épinière, cortex, ...) ne présentent que peu de cellules marquées par \gls{musk}, très loin de ce qui est observable ailleurs.

%Culture cellulaire
Le co-marquage de \gls{musk} avec des marqueurs neuraux et astrocytaire sur des cultures de cellules d'hippocampe nous apprend que le récepteur est à la fois exprimé par des astrocytes (co-marquage avec \gls{gfap}), mais également par des neurones (co-marquage avec \gls{map2}), au niveau des dendrites et potentiellement du soma. Quand on regarde des coupes de cerveau cependant, on n'observe cependant qu'un marquage dans des prolongements astrocytaire, et non dans des dendrites. Le marquage du soma pourrait également être dû à un début de dégénérescence des cellules, qui ont été cultivées et fixées plus longtemps que ce que le laboratoire à l'habitude de faire.

On peut observer dans le cerveau un léger marquage de ce qui ressemble à des corps cellulaires. J'ai tout d'abord pris ce marquage pour du bruit de fond, ou un passage du \acrshort{dapi} dans le canal vert. Cependant, au vu des cultures cellulaires, on peut penser que ce serait au contraire un marquage de\gls{musk} au niveau du soma. Cependant, vu la qualité du marquage et la forte ressemblance avec les coupes contrôles sans anticorps primaires, il n'est pas possible de conclure sur ce fait.

Au microscope confocal, le marquage \gls{musk} est plus faible dans les prolongements \gls{map2}-positif  que les prolongements \gls{gfap}-positif. On peut penser que la protéine est plus faiblement exprimé par les neurones, et donc n'est pas visible en microscopie à épifluorescence en tranche entière à cause du bruit de fond présent, ou bien alors que la densité des dendrites est trop importantes pour qu'un motif particulier ressorte. Une troisième hypothèse concernant les différences de marquages entre coupes et cultures pourrait être la différence maturation des  cellules nerveuses. Les cultures sont issues d'embryon au stade E16 tandis que les coupes de cerveaux sont issues de souris adultes (30 jours environ). Cependant, en observant les coupes d'embryon E18.5 sauvages, on ne voit pas non plus l'apparition d'un marquage \gls{musk} dans les dendrites neuronales.

%Rôle de MuSK
Le rôle crucial des astrocytes dans le fonctionnement de la synapse est de plus en plus mis en avant, au point de parler de synapse "tri-partite" \cite{Araque1999, Perea2009}. Les astrocytes à la synapse vont avoir un rôle principalement dans le recyclage des neurotransmetteurs, en endocytant les molécules présentes dans la fente synaptique. La présence de \gls{musk} à la fois dans des dendrites et dans des prolongements astrocytaires pourrait faciliter la formation de cette synapse tri-partite, si l'on considère que ces deux éléments forme l'ensemble post-synaptique de la synapse. Le rôle de \gls{musk} serait alors identique à son rôle à la \gls{jnm}, et permettrait la formation de la synapse en préparant la dendrite et l'astrocyte à l'arrivée de l'axone, et en permettant la maturation de la jonction entre ces trois éléments.

%Expression de MuSK
Concernant l'expression de \gls{musk}, de nombreux gènes sont exprimés de manière différentes entre les deux hémisphères, notamment entre les deux hippocampes \cite{Moskal2006}. Cette asymétrie Gauche-Droite est un processus essentiel notamment à la formation de la mémoire \cite{Shimbo2018}. L'expression de \gls{musk} suit cette asymétrie, et est plus fortement importante dans l'hippocampe gauche de la souris que dans l'hippocampe droit. On peut supposer que l'expression de \gls{musk} est semblable entre les individus sauvages et mutants, car l'immunomarquage du récepteur est semblable entre les deux types d'individus.

Ce travail est l'un des premiers à fournir un aperçu de la localisation et du rôle de \gls{musk} dans le cerveau. Il a permit de montrer que le récepteur \gls{musk} était localisé dans des régions discrètes du cerveau, et à permis par immunomarquage de confirmer sa présence dans des neurones et astrocytes, qui n'avait alors été observer que par \gls{his} et \gls{wb} chez l'adulte, et par \gls{qpcr} chez des cultures de cellules neurales embryonnaires. J'ai également pu montré que \gls{musk} allait jouer un rôle dans la mise en place de  plusieurs régions de l'hippocampe, au travers notamment de son domaine \gls{crd}. J'ai enfin montré que l'expression du récepteur \gls{musk} était asymétrique entre les deux hippocampes. De nombreux travaux dans le domaine restent cependant nécessaires.