%Marquage, loca musk
Grâce aux co-marquage de \gls{musk} et \gls{gfap}, j'ai pu montré que dans le cerveau, \gls{musk} est exprimé par les astrocytes. Cependant,  un doute reste sur la spécificité du marquage de l'anticorps anti-\gls{musk}. Hessert \emph{et al.} en 2006 ont généré une lignée de souris \gls{musk}/LoxP croisée avec une souris où le gène de la recombinase Cre est sous promoteur de la créatine kinase \cite{Hesser2006a}. Dans cette souris Cre/LoxP, \gls{musk} est inactivé dans les cellules musculaires au cours du développement, à la différence des souris KO. On peut alors imaginer croiser la lignée \gls{musk}/LoxP avec une lignée de souris Cre sous contrôle d'un promoteur nerveux (exprimé à la fois par les neurones et par les cellules glial) pour obtenir un modèle de souris \gls{musk} KO spécifique du cerveau. Cette lignée permettrait d'étudier l'effet du récépteur dans le cerveau, et permettrait également de confirmer la spécificité du marquage.

Un autre moyen de confirmer le marquage de \gls{musk} serait de réaliser un co-marquage avec un autre anticorps dirigé contre le récepteur. J'ai appris récemment l'existence d'un tel anticorps utilisé par l'équipe de Lin Mei, utilisable en immunofluorescence. Le co-marquage de deux anticorps dirigé contre deux régions différentes du récepteur serait un bon indice sur la spécificité du marquage. 

%Culture cellulaire
Grâce au marquage de \gls{musk} sur des cultures de cellules d'hippocampe, on peut observé que le récepteur est à la fois exprimé dans des astrocytes (co-marquage avec \gls{gfap}), mais également dans les dendrites de neurones (co-marquage avec \gls{map2}). Quand on regarde des coupes de cerveau cependant, on n'observe cependant qu'un marquage dans des prolongement astrocytaire, et non dans des dendrites. Il apparaissait au microscope confocal que le marquage \gls{musk} était plus faible quand il colocalisait avec \gls{map2} que quand il colocalisait avec \gls{gfap}. On peut alors penser que la protéine est trop faiblement exprimé par les neurones pour ressortir clairement en microscopie à épifluorescence en tranche entière, ou bien alors que la densité des dendrites est trop importantes pour qu'un motif particulier ressorte. 

Une troisième hypothèse pourrait être la maturation des  cellules nerveuses. Les cultures sont issues d'embryon au stade E16 tandis que les coupes de cerveaux sont issues de souris adultes (30 jours environ). Le profil d'expression des gènes n'est pas le même à ce moment là, d'où une présence de \gls{musk} dans les dendrites des neurones en cultures, marquage qui n'est pas visibles sur les coupes de cerveau. 

%Expression de MuSK
Pour l'expression de \gls{musk}, de nombreux gènes sont exprimés de manière différentes entre les deux hémisphères, notamment entre les deux hippocampes \cite{Moskal2006}. Cette asymétrie Gauche-Droite est un processus essentiel notamment à la formation de la mémoire \cite{Shimbo2018}. \gls{musk} suit asymétrie et apparaît plus fortement exprimé dans l'hippocampe gauche que dans l'hippocampe droit. Cependant, cette tendance ne ressort par chez les souris mutante. Mais le gène chez ces dernière s'est amplifié tard (après le 35\up{ème} cycle). Ces résultats ne sont donc pas en eux-mêmes vraiment interprétable. Il n'est inenvisageable que ce qui ai été amplifié soit dû au bruit de fond. Cependant, on peut voir qu'il y a une différence notable entre les souris sauvages et les souris mutantes. 

On peut donc poser deux hypothèse : Soit effectivement, l'expression de \gls{musk} est fortement réduite chez les mutants, soit la mutation empêche l'appariement des primers sur le gène. Ces primers fournis par Qiagen\texttrademark sont propriétaire et leur séquence exacte n'est pas connue. Cependant, la région où les primers s'accrochent est éloignée de 1000pbs de la région du \gls{crd}, donc cette dernière hypothèse est peu probable.

Il est a noté que bien que l'expression de \gls{musk} soit différents entre les individus sauvages et mutants, le marquage de la protéine était semblable chez les deux populations. 

Sun \emph{et al.} avaient décrit la présence de \gls{musk} dans les astrocytes et les neurones par \gls{qpcr}. Les cellules provenaient d'embryon au stade E18. Ici, j'ai étudié l'expression de \gls{musk} chez des souris agées de 30 jours. Il serait intéressant d'observer l'évolution de l'expression de \gls{musk} au cours du développement, au moyen de \gls{qpcr} réalisés à différents stades embryonnaire et post-nataux.

Cultures
Hypothèse rôle Musk
Tempéré résultats neun : taille effet ? n suffisant ? variations lors coupes/mesures.