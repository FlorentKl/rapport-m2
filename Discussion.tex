%Structure du Cerveau 
J'ai pu montrer durant mon stage que bien que l'organisation globale du cerveau des souris \mcrd restait inchangée, la structure de l'hippocampe allait être modifié, avec la présence d'un rétrécissement de la région CA3 dans la partie caudale du cerveau, ou bien une modification de la taille de la couche granulaire du Gyrus Denté. Cependant, il est possible que j'ai été biaisé lors de mesmesures ou du choix des coupes. Il serait alors intéressant de refaire ces mesures en augmentant le nombre d'individu, qui était de 3 souris souris sauvages et 4 souris mutantes.

%Marquage, loca musk
Grâce aux co-marquage de \gls{musk} et \gls{gfap}, j'ai pu montré que dans le cerveau, \gls{musk} est exprimé dans les astrocytes, qui semblent être de type fibreux, avec de long prolongements lisses. Cependant, un doute persiste sur la spécificité du marquage de l'anticorps anti-\gls{musk}. Hessert \emph{et al.} en 2006 ont généré une lignée de souris \gls{musk}/LoxP croisée avec une souris où le gène de la recombinase Cre est sous promoteur de la créatine kinase \cite{Hesser2006}. Dans cette souris Cre/LoxP, \gls{musk} est inactivé dans les cellules musculaires après la naissance, à la différence des souris KO. On peut alors imaginer croiser la lignée \gls{musk}/LoxP avec une lignée de souris Cre sous contrôle d'un promoteur nerveux (exprimé à la fois par les neurones et par les cellules glial) pour obtenir un modèle de souris \gls{musk} KO spécifique du cerveau. Cette lignée permettrait d'étudier l'effet de la délétion de \gls{musk} sur le comportement, et permettrait également de confirmer la spécificité du marquage astrocytaire.

Il apparaît que les structures marquées par \gls{musk} sont des régions riches en fibres nerveuses et en astrocytes. Les régions denses en corps cellulaire (substance grise de la moëlle épinière, cortex, ...) ne présentent que peu de cellules marquées par \gls{musk}, très loin de ce qui est observable ailleurs.

%Culture cellulaire
Le co-marquage de \gls{musk} avec des marqueurs neuraux et astrocytaire sur des cultures de cellules d'hippocampe nous apprend que le récepteur est à la fois exprimé par des astrocytes (co-marquage avec \gls{gfap}), mais également par des neurones (co-marquage avec \gls{map2}), au niveau des dendrites. Quand on regarde des coupes de cerveau cependant, on n'observe cependant qu'un marquage dans des prolongements astrocytaire, et non dans des dendrites. Au microscope confocal, le marquage \gls{musk} était plus faible quand il colocalisait avec \gls{map2} que quand il colocalisait avec \gls{gfap}. On peut alors penser que la protéine est trop faiblement exprimé par les neurones pour être visible en microscopie à épifluorescence en tranche entière, ou bien alors que la densité des dendrites est trop importantes pour qu'un motif particulier ressorte. 

Une troisième hypothèse pourrait être la maturation des  cellules nerveuses. Les cultures sont issues d'embryon au stade E16 tandis que les coupes de cerveaux sont issues de souris adultes (30 jours environ). Cependant, en observant les coupes d'embryon E18.5 sauvages, on ne voit pas non plus l'apparition d'un marquage \gls{musk} dans les dendrites neuronales.

%Rôle de MuSK
Le rôle crucial des astrocytes dans le fonctionnement de la synapse est de plus en plus mis en avant, au point de parler de syapse "tri-partite" \cite{Araque1999, Perea2009}. La présence de \gls{musk} à la fois dans des dendrites et dans des prolongements astrocytaires pourrait faciliter la formation de cette synapse tri-partite, si l'on considère que ces deux éléments forme l'ensemble post-synaptique de la jonction. Le rôle de \gls{musk} serait alors identique à son rôle à la \gls{jnm}, et permettrait la formation de la synapse en préparant la dendrite et l'astrocyte à l'arrivée de l'axone, et en permettant la maturation de la jonction entre ces trois éléments.

%Expression de MuSK
Pour l'expression de \gls{musk}, de nombreux gènes sont exprimés de manière différentes entre les deux hémisphères, notamment entre les deux hippocampes \cite{Moskal2006}. Cette asymétrie Gauche-Droite est un processus essentiel notamment à la formation de la mémoire \cite{Shimbo2018}. \gls{musk} suit cette asymétrie et apparaît plus fortement exprimé dans l'hippocampe gauche que dans l'hippocampe droit chez les souris \gls{wt}. 
%Cependant, cette tendance ne ressort pas chez les souris mutantes, où l'expression de \gls{musk} est constante, et faible par rapport au taux d'expression du récepteur chez les souris \gls{wt}. Le gène chez les souris \mcrd s'est amplifié assez tardivement (après le 35\up{ème} cycle). Ces résultats ne sont donc pas en eux-mêmes vraiment exploitable. Il n'est pas inenvisageable que ce qui ai été amplifié soit dû au bruit de fond, et non pas à la présence du gène. On peut rejeter un problème d'extraction des \acrshort{arn}, car l'expression du gène rapporteur \gls{26s} est semblable entre les individus sauvages et mutants. 

On peut donc poser deux hypothèse à cette baisse d'expression chez les souris mutantes : soit effectivement, l'expression de \gls{musk} est fortement réduite à cause de la mutation, ou alors la mutation empêche l'appariement des primers sur le gène. Ces primers fournis par Qiagen\texttrademark sont propriétaires et leur séquence exacte n'est pas connue. Cependant, la région où les primers s'accrochent est éloignée d'environ 1000pbs de la région du \gls{crd} : cette dernière hypothèse est ainsi peu probable.

Il est à noter que bien que l'expression de \gls{musk} soit différents entre les individus sauvages et mutants, le marquage de la protéine sur coupes de cerveau est semblable dans les deux populations de souris. 