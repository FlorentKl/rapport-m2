% !TeX program = lualatex
% !TeX encoding = UTF-8
% !TeX spellcheck = fr_FR

%Test 12  12 

\documentclass[12pt, twoside]{report}

%%%%% PACKAGE %%%%%
\usepackage{polyglossia} %Langue
\usepackage{geometry}
\usepackage{fontspec} %Police
\usepackage{unicode-math} %Plante si absent mais formule dans texte
%\usepackage[adobe-garamond]{mathdesign} %Police signe maths %%Fait buguer avec unicode-math
\usepackage{csquotes}
\usepackage[sorting=none,
							backend=bibtex,
							style=ieee,
							block=none,
							minbibnames=3, maxbibnames=9,
							]{biblatex} %A voir + tard
\defbibheading{subbibliography}[\refname]{\section*{#1}}
\usepackage{microtype}
\usepackage{graphicx} %Gestion des images
\usepackage{wrapfig} %Gestion des images, permet mise image et texte cote a cote
\usepackage{caption} %Titre figure mieux géré
\usepackage{subcaption} %idem sous figures
\usepackage{titlesec}
\usepackage{hyperref} %Meilleure gestion pdf
\usepackage{fancyref} 
\usepackage{cleveref} %Ref mieux gérer
\usepackage[	toc, %Apparaît dans table des matières
				nopostdot, %Pas de points a fin acronyms
				nonumberlist,%Pas de numéro de pages
				nogroupskip, %Pas de regroupement par première lettre
				]{glossaries} 
\usepackage{todonotes}
\usepackage{textgreek}
\usepackage{fancyhdr} %Gestion en-tête et pieds de page
\usepackage{xspace} %Gestion espace dans macro
\usepackage{titling} %Utilisation titre dans en-tête
\usepackage[nottoc]{tocbibind} %Bibliographie dans Table des matières
\usepackage[version=4]{mhchem} %Ecriture ion/molécules
\usepackage{parskip}

%test
%%%%% TRUC 
%%%%% INFO %%%%%
\hypersetup{	pdfauthor={Florent KLEE},
	pdftitle={Roles de Wnts et MuSK, un récepteur tyrosine kinase dans le cerveau},
	pdfsubject={Roles de Wnts et MuSK dans le cerveau},
	colorlinks,
	citecolor=black,
	filecolor=black,
	linkcolor=black,
	urlcolor=blue,}

%%%%% PARAM %%%%%
\geometry{	a4paper,
			top=2.5cm,
			bottom=2.5cm,
			left=2.5cm,
			right=2.5cm}
\setmainfont[	Ligatures=TeX,
				Kerning=Uppercase, %Meilleurs intégration Maj-Min
				BoldFont={AGaramondPro-Semibold},
				ItalicFont={AGaramondPro-Italic},
				Numbers=Proportional,]{Adobe Garamond Pro}
\setmathrm{Adobe Garamond Pro}
\setmainlanguage{french}
\setotherlanguage{english}
\settocbibname{Références}
\setlength{\parindent}{3em}

%Redef format chapitre, notamment reduire espace au dessus et dessous de ceux ci
\makeatletter
\titleformat{\chapter}[hang]
{\normalfont\huge\bfseries}{\chaptertitlename~\thechapter : }{17\p@}{\huge}
\titlespacing*{\chapter}{0pt}{10\p@}{13\p@}
\makeatother

\captionsetup[figure]{labelfont={bf},labelformat={default},labelsep=period,name={Figure }} %Affichage Figure x. au lieu de fig x.

\fancypagestyle{plain}{ %Redef style page plain de page avec Chapitre
	\fancyhf{} % clear all header and footer fields
	\fancyhead[C]{\textit{\titredoc}}
	\fancyfoot[C]{\textbf{\thepage}} % except the center
	\renewcommand{\headrulewidth}{0,4pt}
	\renewcommand{\footrulewidth}{0pt}}
\setlength{\headsep}{25pt}

%%%%% COMMANDES / MACRO %%%%%
\newcommand{\blankpage}{\newpage\thispagestyle{empty}\null\newpage} %Création page blanche
\newcommand{\up}[1]{\textsuperscript{#1}} %Permet exposants
\newcommand{\mcrd}{MuSK\textDelta{}CRD\xspace} %Ecrire MuSK-CRD
\newcommand{\titredoc}{Rôles de Wnts et MuSK,\\un récepteur tyrosine kinase dans le cerveau }

\renewcommand{\thefigure}{\arabic{figure}} %Permet "Figure X" au lieu de "Figure Y.X" (Y étant num chapitre, X num figure)

%%%%% GLOSSAIRE %%%%%
\makeglossaries
\loadglsentries{./Glossaire.tex}

%%%%%Ressources biblio %%%%%
\addbibresource{library.bib}

%%%%%%%%%% DOCUMENTS %%%%%%%%%%
\begin{document}
\pagestyle{fancy}
\fancyhf{}
\chead{\textit{\titredoc}}


%%%%% Page de Garde %%%%%	
\begin{center}

%%%%% LOGO %%%%%
\begin{figure}[!h] %!h = en haut ! 
	\begin{minipage}{0.48\textwidth}
		\raggedright %Tout a gauche
		\includegraphics*[height=0.1\textheight]{./Images/Logo_P5.png}
	\end{minipage}%	
	\hfill
	\begin{minipage}{0.48\textwidth}
		\raggedleft %Tout a droite
		\includegraphics*[height=0.1\textheight]{./Images/Logo_CNRS.png}
	\end{minipage}%
\end{figure}
\vspace*{1.5cm}

%%%%% EN TETE %%%%%
Rapport de stage\\
Master 2 Neurosciences\\
Année 2017-2018\\
\vspace{1cm}


%%%%% TITRE %%%%%
\rule{\textwidth}{0.5pt} \\[0.4cm]
{\huge \bfseries \titredoc\\[0.4cm]} %titre du document
\rule{\textwidth}{0.5pt} \\[1.5cm]

%%%%% AUTEURS / SUPERVISEUR %%%%%
Florent KLEE \\
CNRS - UMR8119\\
Équipe Développement et Pathologies des Jonctions Neuromusculaires\\
Université Paris Descartes\\
Encadrante : Claire LEGAY\\
\vspace{1cm}
\missingfigure{Ajout image fluo ?}

%%%%% UMR8119 %%%%%
\begin{figure}[!b] %!b : En bas de la page !
	\centering
	\includegraphics[height=0.1\textheight]{./Images/Logo_UMR8119.png}
\end{figure}

\end{center}\thispagestyle{empty}
\blankpage

%%%%% Remerciement %%%%%
Je tiens en premier lieu à remercier Claire Legay, qui a su m'accueillir chaleureusement dans son équipe, me prodiguer ses conseils, et corriger patiemment ce rapport.

{\setlength{\parindent}{0cm}
Merci également à Pascale Leblanc, sans qui ce stage n'aurait simplement pas été possible, au vu de l'aide qu'elle m'a apportée avec les animaux.

Merci a Fannie Semprez et à Susie Barbeau pour la bonne ambiance dans l'équipe, ainsi que pour leur aide pendant les expériences.

Merci à Alexandre Dobbertin pour son avis et ses conseils durant les Western Blot.

Merci à Damien Carrel pour les cultures d'hippocampes, et son avis sur les marquages.

Merci à Cendra Agulhon pour son aide dans dans l'identification des marquages.

Merci à Laure Strochlic et à Julien Messéant pour les animaux fournies, ainsi que leurs remarques constructives.

Merci au SCM, à Jean-Maurice Petit et à Jennifer Coridon pour leur aide sur la microscopie.

Et merci encore à tout ceux qui ont permis à ce stage de se réaliser.}\thispagestyle{fancy}
\newpage

%%%%% Table des Matières %%%%%
\setcounter{page}{1}
\cfoot{\textbf{\thepage}}
\tableofcontents
\newpage

%%%%% Corps Documents %%%%%
 % !tex root= main.tex
  
\section{Formation de la Jonction Neuromusculaire}
\label{sec:IntroSynapse}
	La \gls{jnm} est une synapse qui permet la transmission nerveuse entre le motoneurone, un neurone dont le corps cellulaire est localisé dans la moëlle épinière, et la fibre musculaire squelettique. La \gls{jnm} est indispensable à la survie de l'organisme, permettant les mouvements volontaires et la respiration. De part sa grande taille et sa facilité d'accès, la \gls{jnm} est depuis longtemps le modèle préférentiel d'étude des synapses du point de vue structural, développemental, et physiologique. Chez les vertébrés, le neurotransmetteur présent à la \gls{jnm} est l'\gls{ach}. 

	L'apposition de l'élément pré-synaptique (axone) sur l'élément post-synaptique (fibre musculaire) requiert au préalable une différenciation post-synaptique qui se manifeste par la présence d'agrégats de \gls{achr} au milieu de la fibre musculaire. L'étape de formation d'agrégats aneuraux de \gls{achr} qui se déroule avant la reconnaissance du muscle par l'axone et l'ancrage de celui-ci sur l'élément post-synaptique se nomme "muscle pre-patterning" \cite{Wu2010a, Gordon2012}. Elle dépend entièrement de la présence de \gls{musk}, un récepteur tyrosine kinase qui va avoir plusieurs rôles dans la formation de la \gls{jnm} : attirer l'axone, stimuler la formation et le remodelage des agrégats de \gls{achr} chez l'embryon ainsi que maintenir la synapse chez l'adulte \cite{Hesser2006}.

	Lors du développement, le cône de croissance de l'axone se dirige vers l'élément post-synaptique (\cref{fig:FormaJNM}). Quand les deux éléments entrent en contacts (au jour embryonnaire E14 chez la souris), des cascades de signalisation sont initiées, ce qui résulte en la différentiation des parties pré- et post-synaptiques \cite{Sanes1999}, avec notamment une redistribution des clusters de \glspl{achr}, qui ne sont plus présent dans les régions extrasynaptiques, mais uniquement concentrés au niveau de la synapse.

	\begin{figure}[h]
		\includegraphics[width=\textwidth]{./Images/formation_jnm.png}
		\caption{Formation de la \gls{jnm}.} 
		\descfig{Figure issue de Burden \emph{et al.} 2018 \cite{Burden2018}.}
		\label{fig:FormaJNM}
	\end{figure}

	Ainsi, l'acteur majeur du développement de la synapse neuromusculaire est donc \gls{musk}, qui sert de plateforme de signalisation à la fois durant les étapes de pré-patterning, puis durant les étapes de mise en place et de maturation de la synapse.
		
\section{Le récépteur \acrshort{musk}, une molécule clef de la synaptogénèse}
\label{sec:IntroMuSK}	

	\begin{wrapfigure}{l}{0.25\textwidth}
		\centering{\includegraphics[width=0.1\textwidth]{./Images/MuSKReceptor.png}}
		\caption{Récepteur \gls{musk}}
		\descfig{Ig : Domaine immunoglobuline, CRD : \emph{Cysteine Rich Domain}, TM : Domaine transmembranaire,TKD : Domaine tyrosine kinase.}
		\label{fig:RMuSK}
	\end{wrapfigure}
	
	\acrfull{musk} est un récepteur découvert dans l'organe électrique de la raie \emph{Torpedo california} \cite{Jennings1993}. L'expression de ce récepteur à d'abord été quantifiée dans les cellules musculaires et localisée au niveau de la \gls{jnm}. \gls{musk} est un récepteur tyrosine-kinase de 98kDa, dans lequel on distingue trois parties : un ectodomaine (partie N-terminale), un domaine transmembranaire, et un domaine cytoplasmique qui porte l'activité kinasique (voir \cref{fig:RMuSK}). 
	
	La partie extracellulaire comporte trois domaines de type \gls{ig}, dont le domaine \gls{ig}1 a récemment été impliqué dans la liaison avec le \gls{lrp}4 \cite{Zhang2011}, ainsi qu'un domaine Frizzled-like, riche en cystéines (\gls{crd}) \cite{Jing2009}.
	
	\gls{musk} possède trois ligands connus : l'Agrine (via \acrshort{lrp}4), un collagène spécifique associé à l'\Gls{ache} appelé \acrshort{colq}, et les \Glspl{wnt}, tous nécessaire au développement complet de la synapse. Un défaut de signalisation de l'un d'entre eux entraîne ainsi des défauts structuraux et/ou fonctionnels de la synapse. L'Agrine joue un rôle prépondérant par rapport aux autres ligands.
	
	%Précédemment à la fin de section 1.
	L'Agrine est le ligand historique de \gls{musk} \cite{Glass1996}, dont une isoforme neuronale est sécrétée par l'axone au contact de la cellule musculaire. Plus récemment, des travaux ont montré que l'Agrine se fixait en fait sur le co-récépteur de \gls{musk} : \gls{lrp}4 \cite{Zhang2008,Kim2008}. Suite à l'activation par l'Agrine de \acrshort{lrp}4, deux complexes \gls{musk}/\gls{lrp}4 vont s'assembler, et cet assemblage tétramérique permettrait une phosphorylation optimale de \gls{musk}, et donc une différenciation de la synapse et de l'agrégation des \gls{achr} \cite{Zong2012}.
	
	La présence de \gls{musk} dans le cerveau a longtemps été ignorée, du fait de sa faible expression dans cette organe, quantifiée dans le passé par Northern Blot, une méthode de détection des \acrshort{arnm} peu sensible. Cependant, de nouvelles techniques, telle que l'\gls{his} ou bien la \gls{qpcr}, ont permis de montrer que le récepteur était bien présent dans le tissu cérébral, principalement au niveau des neurones du cortex, du cervelet, et de l'hippocampe \cite{Garcia-Osta2006, Ksiazek2007}. Le récepteur \gls{musk} semble aussi être exprimé fortement dans les astrocytes \cite{Sun2016}, à des taux jusqu'à 5 fois supérieur à son expression dans les muscles squelettiques.
	
	Bien que sa présence soit prouvée dans le cerveau, le rôle de \gls{musk} dans cette structure reste assez méconnu. 
	
	%WTF a voir ou inserer.
	% où avec son co-récepteur \gls{lrp}4 il régulerait la transmission glutamatergique par l'intermédiaire de la libération d'ATP et une signalisation liée à l'agrine.
	
	\todo{a améliorer}
	Au niveau du \gls{snc}, deux isoformes de \gls{musk} semblent être exprimées \cite{Garcia-Osta2006}. La première isoforme, de 2644pbs, est identique à une isoforme générée par épissage alternatif dans le muscle \cite{Valenzuela1995}, sans qu'aucun rôle ne lui soit connu pour l'instant. La seconde isoforme est plus courte : 2359pbs, et présente une délétion du troisième domaine \gls{ig}. Les deux isoformes présentent une alanine à la position 454 qui remplace une délétion de 8 acides aminé de l'ectodomaine. Une autre isoforme ayant le domaine \gls{ig}3 supprimé serait impliquée dans l'agrégation des \gls{achr} \cite{Hesser1999}.
	
	Grâce à des techniques de knockdown du gène par séquence antisens au niveau de l'hippocampe, il apparaîtrait que la présence de \gls{musk} dans le cerveau serait nécessaire mais non indispensable à la formation de la mémoire à moyen et long-terme \cite{Garcia-Osta2006}. La voie \gls{creb} est une voie impliquée dans la formation de la mémoire au niveau de l'hippocampe \cite{Silva1998, Kandel2012,Kida2014,Ortega-Martinez2015}, qui serait médié par la phosphorylation de \gls{creb} suite a la libération de \acrshort{camp} \todo{par quoi ?}, augmentant son activité transcriptionnelle. Un modèle propose que l'activation de \gls{musk} activerait la cascade de signalisation de \gls{creb}, permettant la consolidation de la mémoire \cite{Garcia-Osta2006} . Ce modèle expliquerait également l'auto-régulation de \gls{musk} \cite{Moore2001}, dont le gène possède dans sa séquence promotrice un élément CRE-like liant \gls{creb} \cite{Kim2005}. De plus, \gls{musk} est nécessaire à la formation de la \gls{ltp} de l'hippocampe \cite{Garcia-Osta2006}.

\section{Les protéines \Acrshortpl{wnt}, ligands de \acrshort{musk}}
\label{sec:IntroWnt}	
	\begin{wrapfigure}{l}{0.4\textwidth}
		\includegraphics[width=0.4\textwidth]{./Images/WntProtein.png}	
		\caption{Structure d'une protéine \Gls{wnt} classique.}
		\descfig{Figure issue de Willert \& Nusse 2012 \cite{Willert2012}. En orange sont représentés les 22 résidus cystéines.}
		\label{fig:WntProt}
	\end{wrapfigure}
	
	Les protéines \glspl{wnt} sont des ligands de \gls{musk}. Outre leur rôle dans la mise en place de la \gls{jnm} médiée par \gls{musk} \cite{Hall2000}, ces protéines sont connues pour leurs rôles prépondérant lors de la neurogénèse et de la mise en place de la connectivité dans les différentes structures du cerveau.
	
	Les \Acrfullpl{wnt} sont des glycoprotéines sécrétées, de 40kDa pour 350 acides aminés, impliquées dans de nombreux processus développementaux tel que l'embryogenèse, la prolifération, la différenciation, la migration cellulaire, ou encore l'apoptose \cite{Miller2002, Willert2012}. 
	%En plus de leurs rôles durant le développement, les \Glspl{wnt} jouent également un rôle à l'age adulte dans la maintenance des tissus adultes. 
	
	La structure des \Glspl{wnt} est complexe, avec  de nombreux ponts disulfures caractéristiques de cette famille de protéines et d'hélices \textalpha{}. Deux modifications post-traductionelles sont courantes : la présence d'un acide palmitoléïque en Ser209 participant à la liaison avec leur récepteur \gls{fz} (voir \cref{fig:WntProt}), et la présence d'un acide palmitique en position Cys77 conservé au cours de l'évolution\cite{Takada2006}. Ces acides gras rendent les protéines \Gls{wnt} très hydrophobes, ce qui a retardé leurs caractérisations.
	
	On connaît actuellement 19 membres de cette famille de protéine chez la souris et chez l'humain. Classiquement, les \Glspl{wnt} se lient sur le domaine \gls{crd} de leur récepteur canonique \gls{fz}, associé aux co-récepteurs \gls{lrp}5 ou 6, mais il existe d'autres récepteurs non canoniques tels que : \gls{ror} \cite{Cadigan2006, Gordon2006, Green2008}, \gls{ryk} \cite{Bovolenta2006, Fradkin2010}, ou bien encore \gls{musk} \cite{Jing2009}, qui possèdent également un \gls{crd}.
	
	Les protéines \glspl{wnt} vont agir aux travers de différentes voies de signalisation dans la cellule : la voie canonique/\textbeta{}-catenin, la voie \gls{pcp}, et d'autres voies indépendantes de la voie \textbeta{}-catenin. La voie canonique fait intervenir \gls{gsk3}, qui participe à la dégradation de la \textBeta{}-catenin et à l'inactivation de la signalisation. 
	
	\emph{In vitro}, il a été montré que plusieurs \Glspl{wnt} interagissaient avec \gls{musk} : \Gls{wnt}2, 3a, 4, 6, 7b, 9a, et 11, qui ont des effets inhibiteurs, stimulateurs, ou sont neutres, sur l'aggragation des \gls{achr} \cite{Strochlic2012, Zhang2012, Barik2014}. Seules \gls{wnt}4, 9a et 11 vont conduire à une dimérisation de \gls{musk} et à son activation (\emph{in vitro}). Ceci est cohérent avec le fait que chez le Poisson-zèbre, l'orthologue de \gls{musk}, \emph{unplugged}, possède aussi un \gls{crd} qui interagit avec des protéines \Glspl{wnt} pour induire l'agrégation de \glspl{achr} \cite{Jing2009, Gordon2012}. \Gls{lrp}4 semble être également nécessaire à l'agrégation des \gls{achr} médié par les \gls{wnt}s \cite{Zhang2012}. De plus, une protéine centrale dans les différentes voie de signalisation des \Glspl{wnt}, \emph{Dishevelled}, a été montré comme pouvant interagir avec \gls{musk} \cite{Luo2002a}.

\section{\acrshort{musk} et \Acrshortpl {wnt} : Contexte de l'étude et but du stage}
\label{sec:Contexte}	
	Dans le but d'étudier le rôle de l'interaction des protéines \Glspl{wnt} et du domaine \gls{crd} de \gls{musk}, l'équipe de C. LEGAY à crée une souris transgénique dont le \gls{crd} était supprimé (\mcrd) \cite{Messeant2015, Messeant2017}. Il a ainsi été montré que le \gls{crd} était nécessaire à la \gls{jnm} à la fois pour sa formation et pour son maintien à l'age adulte, et que les \Glspl{wnt} 4 et 11 participaient activement à la mise en place de la synapse. De plus, un traitement au \gls{licl} (inhibiteur de la \gls{gsk3}) permet une réstauration du phénotype sauvage de la \gls{jnm} \cite{Messeant2015}. Cet effet du \gls{licl} indique que la voie \textbeta{}-catenin est impliqué dans la signalisation \Glspl{wnt}-\gls{musk}. %La voie \gls{pcp} serait également impliqué.
	
	En plus de leurs problèmes musculaires, les souris \mcrd exhibaient des défauts centraux : durant son stage, une étudiante, Bertille SOMON, a montrée que les mutants mâles avaient des blessures importantes au niveau du dos, blessures qui n'étaient pas dues à des comportements d'agressivité entre souris. De plus, une analyse comportementale a été réalisée en collaboration avec le groupe du Dr LANFUMEY (Centre de Psychiatrie et Neurosciences, Paris), et le \gls{nor} a révélé que les souris mutantes souffrent d'un déficit de la mémoire intermédiaire.
	
	\todo{Wnt MEMOIRE !!!!}
	Comme \gls{musk} est exprimé dans le cerveau adulte, principalement au niveau de l'hippocampe \cite{Garcia-Osta2006}, et que cette structure joue un rôle prépondérant dans la formation de la mémoire intermédiaire, l'objectif de mon stage a été d'explorer le rôle de l'interaction de \gls{musk} et des \Glspl{wnt} dans le cerveau, utilisant pour cela les souris \mcrd. Dans cette perspective, j'ai posé plusieurs questions : Est-ce que la structure du cerveau est affectée chez le mutant, quelles sont les cellules exprimant \gls{musk}, et quel est le niveau d'expression de \gls{musk} et \mcrd dans le cerveau ? Par ailleurs, j'ai également cherché à comprendre l'origine des défauts de comportement chez le mâle mutant.
	
	%\todo{wnt mémoire ???}
\chapter{Matériel et Méthodes}

\section{Animaux et Prélèvements}
\label{sec:AnimEtPrelev}
	Les souris utilisées sont issues d'une lignée hétérozygote issue d'une fond génétique C57BL/6 mixte \cite{Messeant2015, Messeant2017}. Les souris mutantes \mcrd et \gls{wt} testées sont issues de même portée. 
	Afin de prélever les cerveaux, les souris ont été euthanasiées à l'aide d'une injection péritonéale de Pentorbital. Une injection péristaltique de \gls{pfa} 4\% est réalisée pour fixé les tissus. Le cerveau est retiré et post-fixé dans du \gls{pfa} 4\% pendant 1 heure, puis transféré dans une solution de sucrose 30\% à 4°C jusqu'à utilisation.
	
\section{\acrshort{qpcr}}
\label{sec:qPCR}
	Les hippocampes droits et gauches sont prélevés du cerveau et immédiatement plongés dans 800µl de TRIzol\textregistered, broyés mécaniquement puis stockés à -20°C. L'extraction de l'\gls{arn} se fait grâce au RNeasy Protect Mini Kit de Qiagen\textregistered. Le traitement DNAse, puis la \gls{rtpcr}, se font grâce au RT\up{2} First Strand Kit de Qiagen\textregistered. La \gls{qpcr} se réalise avec SYBR Green/ROX qPCR MM (ThermoFisher\textregistered). Les primers utilisés sont dirigés contre \gls{musk} et \acrshort{26s}.
	
\section{Marquages}
\label{sec:Marquages}
	Les cerveaux sont sectionnés à une épaisseur de 50µm au microtome. Une coloration de Nissl est réalisée sur une coupe sur 3, montée sur lame avec de la gélatine. Après séchage à l'air libre pendant 24 heures, les lames vont être déshydratées et dégraissées avec des bains successifs dans de l'éthanol 70°, 95°, 100° et de xylène. Les lames vont être ensuite réhydratée avec des bains d'éthanol de concentration décroissante avant d'être plongées dans le colorant de Nissl (solution de Crésyl Violet).  Les lames sont ensuite  à nouveau déshydratées et immergées dans un bain de xylène. Le montage se fera dans du milieu Eukitt\textregistered.
	
	Pour l'immunomarquage, une coupe sur 6 centrée sur l'hippocampe est choisie. Les coupes flottantes sont incubés 1 heure à températures ambiante dans un sérum de blocage (\gls{pbs}, Sérum de chèvre 5\%, \gls{bsa} 3\%, Triton 0.2\%), puis sont incubés dans 500µl de solutions de blocage avec anticorps pendant une nuit à 4°C avec agitation. Le lendemain, les coupes sont rinçées 3 fois 7 minutes dans une solution de rinçage/blocage (\gls{pbs}, \gls{bsa} 2,5µg/µl et Triton 1,5\%), puis incubés 1h45 avec l'anticorps secondaire dans le noir à température ambiante, puis 10 minutes avec du \gls{dapi}, sous agitation. Les lames sont à nouveau rincées 3 fois 7 minutes dans la solution précédente. Les lames sont montés dans du milieu de montage Mowiol ou FluoroMount-G.
	
	Les anticorps primaires utilisés sont les suivants : Anti-\acrshort{musk} (polyclonal de lapin, ABS549, Millipore-Merck),  Anti-\acrshort{neun} (polyclonal de lapin, ab104225, Abcam), Anti-\acrshort{gfap} (polyclonal de souris, MAB360, Millipore-Merck). Pour les anticorps secondaires, les suivants ont été utilisés : Anti-rabbit-AlexaFluor\textregistered488 (polyclonal de chèvre, A11008, InVitroGen), Anti-mouse-Cy3 (polyclonal de chèvre, 115-165-003, Jackson ImmunoResearch).
	
\section{Acquisition d'images et Statistiques}
\label{sec:ImagesStats}
	Les images ont été pris avec une microscope Epifluo Nikon Eclypse TE-2000E équipé d'une caméra Photometrics CoolSNAP HQ2 (Fluorescence), et d'une caméra Qimaging Color Retiga 2000R (Lumière blanche).
	Les données ont été mesurées sur images grâce au logiciel ImageJ \cite{Schneider2012}.
\section{Comportement}
\label{sec:Comportement}
	Concernant les blessures préalablement observées sur les souris \mcrd mâles lors du stage de B. SOMON, elles n'ont pas pu être reproduites durant mon stage. Les animaux (mutants comme hétérozygotes) ont cependant été qualifiés de "plus sensible à leurs environnement que la normale" et de "un peu bizarre" par les animaliers en charges de ceux-ci. Comme durant le stage de B. SOMON, les animaux n'étaient pas situés dans la même animalerie qu'actuellement (zone sanctuaire de l'ICM et zone EOPS de l'animalerie de Paris Descartes), on peut penser que les souris étaient soumises à un stress environementale plus important, et adoptaient en réponse à ce stress un comportement d'automutilation, potentiellement accentué chez le mâle.

\section{Structure du cerveau}
\label{sec:NisslResultat}
	La coloration de Nissl, ici à base de Crésyl violet, est un marquage classique du tissu nerveux avec une molécule basique, qui marque particulièrement l'acide nucléique (\acrshort{adn} et \acrshort{arn}) des cellules car basophile (ces structures au microscope prennent le nom de "Corps de Nissl'). Ce marquage est important sur les neurones, riches en \gls{re} rugueux, et donc avec une grande quantité d'\acrshort{arn}. Cette coloration permet ainsi de mettre en évidence les motifs des différentes structures cellulaires au sein du tissu nerveux.

	Comme les protéines \gls{wnt} sont généralement impliquées dans le développement et l'organisation du cerveau, l'ont peut se demander si la suppression du domaine de liaisons de ces protéines sur le récepteur \gls{musk} allait entraîner des modifications importantes du développement du système nerveux.

	Afin de visualiser si cette mutation \mcrd altérait l'organisation structurale du cerveau, une coloration de Nissl a donc été réalisée sur 4 individus : 2 souris mutantes et 2 souris sauvages (1 mâle et 1 femelle pour chaque groupe) (\cref{fig:NisslResultat}). Ce marquage a permis de mettre en évidence que la mutation de \gls{musk} ne modifiait pas de manière visible l'organisation du cerveau, à la fois chez les souris femelles (\cref{fig:FemWTNissl,fig:FemMutNissl}) et mâles (\cref{fig:MaleWTNissl,fig:MaleMutNissl}). 

	\begin{figure}[h] %Figure Nissl Résultats
		\begin{center}
			\begin{subfigure}[h]{0.49\textwidth}%F437 WT Nissl 33 l2.tif
				\caption{}
				\label{fig:FemWTNissl}
				\includegraphics[width=\textwidth]{./Images/Nissl/FemWT.jpg}
			\end{subfigure}
			\begin{subfigure}[h]{0.49\textwidth}%F435 Mut Nissl 38.tif
				\caption{}
				\label{fig:FemMutNissl}
				\includegraphics[width=\textwidth]{./Images/Nissl/FemMut.jpg}
			\end{subfigure}
			\begin{subfigure}[h]{0.49\textwidth}%M2 WT Nissl #031.tif
				\caption{}
				\label{fig:MaleWTNissl}
				\includegraphics[width=\textwidth]{./Images/Nissl/MaleWT.jpg}
			\end{subfigure}
			\begin{subfigure}[h]{0.49\textwidth}%M442 Mut Nissl lame 3 36.tif
				\caption{}
				\label{fig:MaleMutNissl}
				\includegraphics[width=\textwidth]{./Images/Nissl/MaleMut.jpg}
			\end{subfigure}
		\end{center}
		\caption{Pas de modifications structurelles observées chez les mutants.}
		\descfig{Coloration de Nissl sur : \subref{fig:FemWTNissl} Femelle sauvage, \subref{fig:FemMutNissl} Femelle mutante, \subref{fig:MaleWTNissl} Mâle sauvage, \subref{fig:MaleMutNissl} Mâle mutant. Images représentatives de coupes réalisées au niveau de l'hippocampe. Âge moyen des souris : 30 jours. Barre d'échelle : 2 mm.}
		\label{fig:NisslResultat}
	\end{figure}
	%\FloatBarrier

\section{Immunomarquage}
\label{sec:IHC}

	\subsection{\acrshort{neun}}
	\label{ssec:neun}
	\Acrshort{neun} est un marqueur du  corps cellulaire et des noyaux de presque tout les neurones, couramment utilisé \cite{Guselnikova2015}. Ce marquage à été utilisé afin de vérifier la distribution et le nombre des neurones au niveau de l'hippocampe des souris \mcrd était perturbée ou non. Pour cela, des mesures de l'épaisseur de la couche pyramidale de les hippocampes Droit et Gauche ont été mesurés dans trois régions différentes (CA1, CA3 et Gyrus Denté) sur trois coupes par individus.
	
	
	
	\subsection{\acrshort{musk} et \acrshort{gfap}}
	\label{ssec:musk}
	Après avoir observé l'organisation neuronale du cerveau de souris, j'ai voulu observé les régions du cerveau qui exprimaient \gls{musk}. Pour cela, j'ai eu recours à des techniques d'immunomarquages. L'anticorps utilisé est un anticorps polyclonal de lapin, ayant comme immunogène pour sa création un protéine recombinante du domaine extracellulaire de \gls{musk}.
	
	Malgré le fait que le niveau d'expression de \gls{musk} soit considéré comme faible dans le tissu nerveux, j'ai quand même pu observé un marquage de cette protéine dans diverses régions discrètes du cerveau : Hippocampe (couche radiaire et moléculaire principalement), Corps calleux, Habenula, Fasciculus retroflexus, Capsule interne, Noyau caudé, 3ème ventricule ventrale, Cervelet principalement (\cref{fig:ImmunoMusk}). 
	
	\begin{figure}[h] %Figure Immuno MuSK Résultats
		\begin{center}
			\includegraphics[width=\textwidth]{./Images/Immuno/Musk/loca_MuSK.jpg}
		\end{center}
		\caption{Localisation du marquage de \gls{musk} dans le cerveau.}
		\descfig{Immunomarquage de \gls{musk} sur : Tranche de cerveau entière.%
				 alv : alveux hippocampus, cc : corps calleux, cp : pédoncule cérébral, ec : capsule externe, fr : fasciculus retroflexus, hp : hippocampe, or : stratum oriens de l'hippocampe. Barre d'échelle : 2 mm. %
			 	}
		\label{fig:ImmunoMusk}
	\end{figure}
	
	Dans ces régions, le marquage prenait la forme de prolongement cellulaires, qui rappelait ce que l'on peut observer après marquage d'astrocytes. Afin d'étudier cette hypothèse, un co-marquage entre \gls{musk} et \gls{gfap}, un marqueur des astrocytes a été réalisé.
	
	Pour réaliser le marquage de \gls{gfap}, deux anticorps différents ont été utilisés. Le premier, fourni d'abord par l'équipe de C. AGULHON, provient de chez Millipore-Merck (\cref{table:Ac}, réf. MAB360) et fonctionne correctement. Le second anticorps, provenant de chez Abcam (\cref{table:Ac}, réf. ab4648) et commandé suite à une indisponibilité du premier anticorps, s'est révélé inefficace à différentes concentrations testées ($1{:}100$, $1{:}50$). Le premier anticorps a pu heureusement être à nouveau commandé pour la suite des expériences.
	
	Suite au co-marquage de \gls{musk} et \gls{gfap}, on peut observer une colocalisation des deux marqueurs, ce qui semblerait indiqué que \gls{musk} est exprimé par les astrocytes.
	
	%Images Coloc. MuSK GFAP
	\begin{figure}[h]
		\begin{center}
			\begin{subfigure}[h]{0.49\textwidth}
				\caption{}
				\label{fig:ColocMuSK}
				\includegraphics[width=\textwidth]{./Images/Immuno/Musk/MuSK-GFAP/M439_Mut_MuSK.jpg}
			\end{subfigure}
			\begin{subfigure}[h]{0.49\textwidth}
				\caption{}
				\label{fig:ColocGFAP}
				\includegraphics[width=\textwidth]{./Images/Immuno/Musk/MuSK-GFAP/M439_Mut_GFAP.jpg}
			\end{subfigure}
			\begin{subfigure}[h]{0.49\textwidth}
				\caption{}
				\label{fig:ColocMuSK&GFAP}
				\includegraphics[width=\textwidth]{./Images/Immuno/Musk/MuSK-GFAP/M439_Mut_MuSK_GFAP.jpg}
			\end{subfigure}
		\end{center}
		\caption{Le marquage de \gls{musk} et \gls{gfap} colocalise.}
		\descfig{Marquage de \gls{musk} (vert) et de \gls{gfap} (rouge) et \acrshort{dapi} (bleu) au niveau de la région CA1 de l'hippocampe. Le marquage de \gls{musk} va colocaliser avec \gls{gfap} dans toutes les régions du cerveau observées.
				\subref{fig:ColocMuSK} : Marquage de \gls{musk}.
				\subref{fig:ColocGFAP} : Marquage de \gls{gfap}.
				\subref{fig:ColocMuSK&GFAP} : Superposition de \subref{fig:ColocMuSK} et \subref{fig:ColocGFAP}.
				Barre d'échelle : 50µm.}
		\label{fig:colocalisation}
	\end{figure}
	
	Cependant, le marquage observé de \gls{musk} ne ressemble pas à ce que l'on peut observer au niveau de la \gls{jnm}.  On peut donc émettre un doute sur la spécificité du marquage. Comme il n'existe pas vraiment d'autres anticorps dirigés contre \gls{musk} utilisables en \gls{ihc}, j'ai dû essayé d'autres moyen afin de tenter de confirmer la spécificité du marquage. Tout d'abord, un co-marquage \gls{musk}/\acrshort{gfap} à été réalisé sur des sections de cerveau d'embryon de souris \gls{musk} KO (stade E18.5), la mutation étant létale après la naissance pour cause de défaillance respiratoire (\cref{fig:MuskEmbryon}). Sur seulement une coupe d'embryon WT sur cinq, un léger marquage ressemblant à celui observer chez les adultes est présent (\cref{fig:MuskE5WT,fig:MuskE5Marquage}), et aucun marquage n'est présent chez les animaux KO (\cref{fig:MuskE1KO}). Cela ne suffit pas à confirmer la spécificité du marquage de \gls{musk}. Pour tenter alors de confirmer la présence de \gls{musk} dans le cerveau, une immunoprécipitation est alors envisagée.
	
	%Images Embryon
	\begin{figure}[h]
		\begin{center}
			\begin{subfigure}[h]{0.329\textwidth}
				\caption{}
				\label{fig:MuskE5WT}
				\includegraphics[width=\textwidth]{./Images/Immuno/Musk/Embryon/E5WT_50um_500px_df.jpg} 
			\end{subfigure}
			\begin{subfigure}[h]{0.329\textwidth}
				\caption{}
				\label{fig:MuskE5Marquage}
				\includegraphics[width=\textwidth]{./Images/Immuno/Musk/Embryon/E5_WT_MuSK_500px_Zoom_10um.jpg}
			\end{subfigure}
			\begin{subfigure}[h]{0.329\textwidth}
				\caption{}
				\label{fig:MuskE1KO}
				\includegraphics[width=\textwidth]{./Images/Immuno/Musk/Embryon/E1KO_50um_500px_df.jpg}
			\end{subfigure}
		\end{center}
		\caption{Pas de marquage de \gls{musk} sur embryon E18.5 WT et KO.}
		\descfig{Marquage de \gls{musk} et \gls{gfap} sur des embryons (stade E18.5) de souris \gls{musk} KO (n = 2) et WT (n = 1). %
				\subref{fig:MuskE5WT} : Embryon WT. Sur une coupe, un marquage ressemblant à celui de \gls{musk} observé chez les adultes était présent (flèches blanches). %
				\subref{fig:MuskE5Marquage} : Agrandissement du marquage de \gls{musk} encadré en \subref{fig:MuskE5WT}. %
				\subref{fig:MuskE1KO} : Embryon KO. Le marquage observé (triangles blanc) n'est pas spécifique. %
				Images représentatives issues de la région du fornix dorsal. Barre d'échelle : 50µm \subref{fig:MuskE5WT} et  \subref{fig:MuskE1KO}, 10µm \subref{fig:MuskE5Marquage}. 
				}
		\label{fig:MuskEmbryon}
	\end{figure}
	
	(Voir si anticorps Abcam ab58697 ab218245 ab5510 ab92950 ab228488 ont été testés.)

\section{Immunoprécipitation}
\label{sec:IPresultat}
	Afin de confirmer la détection de \gls{musk} dans diverse région du cerveau, une immunoprécipitation suivie d'un \gls{wb} a été réalisée sur trois structures : l'hippocampe, le cervelet et le cortex de trois souris C57Bl/6 sauvage. Un \gls{wb} de \gls{musk} a déjà été décrit sur des diverses structures du cerveau \cite{Garcia-Osta2006}, mais à partir d'extrait total. Durant son stage, B. SOMON a tenté de réaliser un \gls{ip} suivi d'un \gls{wb}, mais sans obtenir de résultats probant. Ici, la principale modification apportée au protocole précédemment utilisée par cette étudiante est l'utilisation de tampon RIPA (adjonction de déoxycholate de sodium et de \acrshort{sds} dans le tampon) qui permet une meilleure lyse et une meilleure préservation des protéines lors de l'extraction.
	
	Afin de vérifier si la technique fonctionnait, j'ai tout d'abord prélevé l'hippocampe, le cervelet et le cortex de trois souris C57Bl/6 puis réalisé une \gls{ip} de \gls{musk} dessus, qui a ensuite été migré dans un gel pour un \gls{wb}. tout d'abord, comme l'anticorps utilisé pour l'\gls{ip} et la révélation était le même, un nouvel anticorps secondaire ciblant uniquement les chaines légères a été utilisé, afin d'éviter d'avoir trop de bruit. Cependant, rien ne fut révélé, même le témoin positif était très faible, malgré un temps d'exposition important (supérieur à 30 minutes) (\cref{fig:WB-anti-LC}). La membrane à été deshybridées puis ré-incubée avec de l'anti-\gls{musk} ainsi qu'un autre anticorps secondaire dirigé contre les \glspl{ig} de lapin, qui à pour défaut d'être plus bruité (bande à 75kDa correspondant aux anticorps non couplés lors de l'\gls{ip}). Surprenamment, des bandes de taille attendue (110kDa) ont été révélée dans l'hippocampe et le cervelet (\cref{fig:WBbon}, flèches noires), alors que dans le cortex une bande d'intensité plus faible semblait être présente (\cref{fig:WBbon}, tête de flèche noire), ce qui pourrait correspondre au fait qu'il n'y ait pas dans cette région du cerveau de marquage de \gls{musk} par \gls{ihc}. 
	
	Comme la taille des bandes ne correspondaient pas tout à fait à celle du témoin positif (extrait de cellules transfectées HEK293, à 130kDa, \cref{fig:WBbon}, flèche blanche), l'expérience à été refaite avec des souris \mcrd et sauvages, issues de la même lignée. En effet, suite à la mutation, la protéine \gls{musk} à un poids moléculaire de ~80kDa : si c'est bien \gls{musk} qui est détécté, on devrait alors avoir un décalage entre les individus. Cette fois-ci cependant, l'expérience ne s'est pas montrée concluante, rien n'a été révélé (\cref{fig:WB1erechec}). Afin de confirmer ce résultat, l'expérience a finalement été tentée une troisième fois avec comme témoin positif les extraits protéiques de la première expérience (cervelet et cortex, plus assez d'extraits d'hippocampe pour recommencer). Cette fois encore, rien n'a été révélé (\cref{fig:WBpasbon}).
	
	%Western Blot	
	\begin{figure}[h]
		\begin{center}
			\begin{subfigure}[h]{0.49\textwidth}
				\caption{}
				\label{fig:WB-anti-LC}
				\includegraphics[width=\textwidth]{./Images/WB/@LC_MuSK_30'.jpg} %Gel anti light chain
			\end{subfigure}
			\begin{subfigure}[h]{0.49\textwidth}
				\caption{}
				\label{fig:WBbon}
				\includegraphics[width=\textwidth]{./Images/WB/2018-04-09.jpg} %Gel Bon :)
			\end{subfigure}
			\begin{subfigure}[h]{0.49\textwidth}
				\caption{}
				\label{fig:WB1erechec}
				\includegraphics[width=\textwidth]{./Images/WB/2018-04-19.jpg} %Gel pas bon :(
			\end{subfigure}
			\begin{subfigure}[h]{0.49\textwidth}
				\caption{}
				\label{fig:WBpasbon}
				\includegraphics[width=\textwidth]{./Images/WB/2018-05-03.jpg} %Gel pas bon :(
			\end{subfigure}
		\end{center}
		\caption{Une \gls{ip} ne permet pas de confirmer la présence de \gls{musk} dans différentes parties du cerveau.}
		\descfig{%
			\subref{fig:WB-anti-LC} : \gls{ip} suivie suivie de \gls{wb} sur 3 souris C57BL/6 révélée avec un anticorps dirigée contre les chaînes légères d'\gls{ig} de lapin. Seule la bande témoin au alentours de 130kDa est révélé. Exposition : 30 minutes. %
			\subref{fig:WBbon} : \gls{ip} suivie de \gls{wb} sur 3 souris C57BL/6. Des bandes sont révélés aux alentours de 110kDa pour les régions de l'hippocampe et du cervelet (flèches noires). Une bande de faible intensité semble être présente pour le cortex (tête de flèche noire). Témoin positif : 15µL d'extrait cellulaire de cellules HEK293 transfectées, bande à 130kDa (flèche blanche). Exposition : 3 minutes. %
			\subref{fig:WB1erechec} : \gls{ip} suivie de \gls{wb} sur 3 souris \mcrd (Mut) et 3 souris sauvages (WT) issues de la même souche. Exposition : 3 minutes.
			\subref{fig:WBpasbon} : \gls{ip} suivie de \gls{wb} sur 3 souris \mcrd (Mut) et 3 souris sauvages (WT) issues de la même souche. Aucune bande n'est révélée, avec ou sans anticorps durant \gls{ip}. Rien n'est révélé non plus chez le témoin positif : Extrait de cervelet et cortex issus de la première expérience. Hipp : Hippocampe, Ct : Cervelet, Cx : Cortex, + : Témoin positif, Ac- : Sans anticorps lors de l'\gls{ip}, Mut : Mutant, \acrshort{wt} : Sauvage. Poids moléculaire de \gls{musk} attendu : 110kDa. Exposition : 3 minutes. %
			}
		\label{fig:WBResultat}
	\end{figure}

\section{Expression de \gls{musk}}
\label{sec:ExpressionMuSK}
	Il est intéressant de connaître le niveau d'expression de la protéine \gls{musk} dans le cerveau, afin de voir si la mutation allait impactée l'expression de la protéine. Pour cela, j'ai eu recours à la technique de \gls{qpcr} sur trois structures différentes : le cervelet, et l'hippocampe gauche ou droit. En effet, des résultats très préliminaire dans le laboratoire chez des 2 souris de souche C57Bl/6 (1 mâle et 1 femelle) avait montrée une légère différence d'expression selon le coté du cerveau étudié. De plus, on retrouvait également une différence entre des individus de sexe opposé. Il était donc nécessaire de reprendre cette technique chez un plus grand nombre d'animaux afin de confirmer ou non ces tendances.

%Structure du Cerveau 
J'ai pu montrer durant mon stage que bien que l'organisation globale du cerveau des souris \mcrd est normale, la structure de l'hippocampe est modifiée, avec le rétrécissement de la région CA3 dans la partie caudale du cerveau, ou bien une modification de la taille de la couche granulaire du Gyrus Denté (diminution de la zone suprapyramidale dans la coupe rostrale, augmentation de la zone suprapyramidale et diminution de la zone infrapyramidale dans la coupe caudale).Ces résultats restent à confirmer en augmentant le nombre d'individu (3 souris souris sauvages et 4 souris mutantes).

%Marquage, loca musk
Durant mon stage, j'ai montré que le récepteur \gls{musk} était exprimé à la fois par les astrocytes, et par les neurones, grâce à de l'immunomarquage sur des coupes de cerveaux, et sur des cultures cellulaires de neurones. Dans les coupes de cerveaux, le marquage observé colocalise avec un marquage de la \gls{gfap}, marqueur des astrocytes. D'après leur morphologie, les astrocytes visualisés sont de type fibreux. Le même type de marquage est observé dans la moëlle épinière. Mes marquages montrant que le récepteur est présent dans des régions discrètes du cerveau. Les différentes régions visualisées par ce marquages sont des régions riches en fibres nerveuses, à la différence de régions plus denses en neurones (Cortex, Substance grise de la moelle épinière...) où ce marquage astrocytaire est beaucoup moins présent. La forme des astrocytes était différente entre les coupes et les cultures cellulaires, mais cela doit être dû à la perte de l'organisation spatiale des cultures cellulaires, et non à une autre populations de cellules.

Cela tranche avec ce qui avait été décrit précédemment grâce à une \acrlong{his} contre \gls{musk}, où les auteurs indiquaient que presque toutes les zones du cerveaux étaient marquées \cite{Garcia-Osta2006}. Les auteurs se sont concentrés sur l'étude des neurones, et non pas décrit la présence de marquage dans des cellules gliales.

Le marquage de cultures cellulaires d'hippocampe avec \gls{musk} et \gls{map2} a permis de montrer que le récepteur était également exprimé par les neurones dans les dendrites. Ce marquage est cohérent avec l'observation par \acrlong{his} de la présence de \gls{musk} sur des tranches de cerveau ou de cultures primaires de neurones d'hippocampes \cite{Garcia-Osta2006}. Cependant, on ne retrouve pas ce marquage sur les coupes de cerveaux, possiblement a cause de de la faible intensité du marquage qui se retrouverait noyer dans le bruit de fond. Le marquage observé dans les prolongements neuronaux sur les cultures cellulaires est moins intense que celui observé dans les astrocytes. Le niveau d'expression par les neurones de \gls{musk} serait donc moins important que dans les astrocytes, ce qui avait déjà été décrit par \gls{qpcr}, bien que dans des cultures cellulaires issues de jeunes souriceaux (stade E18 ou P2-P3 en fonction du type cellulaire étudié) et non dans des cellules plus matures \cite{Sun2016}. 

On peut également observer en microscopie confocale sur les cultures cellulaires un marquage somatique des neurones (co-localisation avec \gls{map2}). Au niveau des coupes, on peut également observer ce type de marquage léger, que j'ai tout d'abord considéré comme du bruit de fond. Cependant, au vu de la qualité du marquage et des coupes contrôles (sans anticorps secondaires) où de telles structures semblent également être présentes, il n'est pas possible de conclure sur ce fait.

Pour le marquage des dendrites neuronaux, plusieurs hypothèses peuvent rendre compte des différences de marquages entres coupes et cultures. Par exemple, le stade de différenciation des cellules pourrait être différents qui est différents \emph{in vivo} et \emph{in vitro}. Les coupes de cerveaux sont issues d'animaux âgés de 30 jours, tandis que les cultures cellulaires sont issues d'embryon au stade E16 et cultivés 12 jours. Cependant, en observant les coupes d’embryon E18.5 sauvages, on ne voit pas non plus l’apparition d’un marquage MuSK dans les dendrites neuronales. Une autre hypothèse pourrait être la représentativité anormale de certains types cellulaires. A la base, les cultures cellulaires ont été faites afin de favoriser la présence de neurones, ce qui pourrait modifier le profil d'expression de \gls{musk}. Enfin, le marquage du soma pourrait être également dû à un début de dégénérescence des cellules lié aux conditions de cultures.

Les astrocytes sont un élément crucial de la synapse, que l'on qualifie de "tri-partite" \cite{Araque1999, Perea2009}. On pourrait penser que \gls{musk} va jouer le même rôle qu'à la \gls{jnm}, et permettre la mise en place de l'élément post-synaptique (ici le prolongement de l'astrocyte et le dendrite neuronal). Cependant, on observerait un marquage ponctiforme correspondants au synapse, et non le marquage continu observé. Comme \gls{musk} semble être présent dans toute la cellule astrocytaire, et dans les dendrites et soma des neurones, on pourrait imaginer différents rôle du récepteur. Ce dernier pourrait être impliqué dans la neuroprotection, comme senseur extracellulaire, 

Concernant l'expression de \gls{musk}, de nombreux gènes sont exprimés de manière différentes entre les deux hémisphères, notamment entre les deux hippocampes \cite{Moskal2006}. Cette asymétrie Gauche-Droite est un processus essentiel notamment à la formation de la mémoire \cite{Shimbo2018}. \gls{musk} est donc un nouvel exemple d'expression asymétrique, en étant plus fortement exprimé dans l'hippocampe gauche de la souris que dans l'hippocampe droit. On peut supposer que l'expression de \gls{musk} est semblable entre les individus sauvages et mutants, car l'immunomarquage du récepteur est semblable entre les deux types d'individus.

Ce travail est l'un des premiers à fournir un aperçu de la localisation de la protéine \gls{musk} dans le cerveau. Il a permis de montrer que le récepteur \gls{musk} était localisé dans des régions discrètes du cerveau, et par immunomarquage de confirmer sa présence dans des neurones et astrocytes, qui n'avait alors été observer que par \gls{his} et \gls{wb} chez l'adulte, et par \gls{qpcr} dans des cultures de cellules neurales embryonnaires. J'ai également pu montré que \gls{musk} pourrait jouer un rôle dans la mise en place de  plusieurs régions de l'hippocampe, au travers notamment de son domaine \gls{crd}. J'ai enfin montré que l'expression du récepteur \gls{musk} était asymétrique entre les deux hippocampes. De nombreux travaux dans le domaine restent cependant nécessaires.
Durant mon stage, j'ai pu montré que \gls{musk} était exprimé dans des régions discrètes du cerveau, et que le récépteur colocalisait avec \gls{gfap}, un marqueur des astrocytes.

La prochaine étape serait de réaliser des tests comportementaux sur les souris, en présence de \gls{licl} qui à été montré comme pouvant rétablir un phénotype sauvage au niveau de la \gls{jnm} \cite{Messeant2017}. Ces tests pourraient être réalisés en partenariat avec une plateforme basée à l'ICM dédiée au comportement.

%%%%% Inclusion Glossaire %%%%%
\printglossary[	title=Abbréviations, 
				toctitle=Abbréviations, 
				type=\acronymtype]

%%%%% Inclusion Bibliographie %%%%%
\printbibliography[heading=bibintoc, title=Références]

\end{document}