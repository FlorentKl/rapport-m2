 % !tex root= main.tex
 
 \todo{A FAIRE : LIAISON ENTRE LES SECTION}
 
\section{Formation de la Jonction Neuromusculaire}
\label{sec:IntroSynapse}
	La \gls{jnm} est une synapse appartenant au \Acrfull{sn}, qui permet la transmission nerveuse entre le motoneurone et la fibre musculaire squelettique. La \gls{jnm} est indispensable à la survie, permettant les mouvements volontaires et la respiration. De part sa grande taille et sa facilité d'accès, la \gls{jnm} est depuis longtemps le modèle préférentiel d'étude des synapses du point de vue structural, développemental, et physiologique. Chez les vertébrés, le neurotransmetteur présent à la \gls{jnm} est l'\gls{ach}. 

	L'apposition de l'élément pré-synaptique (axone) sur l'élément post-synaptique (fibre musculaire) requiert au préalable une différenciation post-synaptique qui se manifeste par la présence d'agrégats de \gls{achr} au milieu de la fibre musculaire, agrégats qui commencent à se former avant l'arrivée de l'axone \cite{Wu2010a, Gordon2012}. Cette étape de formation d'agrégats aneuraux de \gls{achr} qui se déroule avant la reconnaissance et l'ancrage de l'axone sur l'élément post-synaptique se nomme "muscle pre-patterning". Elle dépend entièrement de la présence de \acrshort{musk}, un récepteur tyrosine kinase qui va avoir plusieurs rôles dans la formation de la \gls{jnm} : attirer l'axone, stimuler la formation et le remodelage des agrégats de \gls{achr} chez l'embryon ainsi que maintenir la synapse chez l'adulte.

	Lors du développement, le cône de croissance de l'axone se dirige vers l'élément post-synaptique (\cref{fig:FormaJNM}). Quand les deux éléments entrent en contacts (au jour embryonnaire E14 chez la souris), des cascades de signalisation sont initiées, ce qui résulte en la différentiation des parties pré- et post-synaptiques \cite{Sanes1999}, avec notamment une redistribution des clusters de \glspl{achr}, qui ne sont plus présent dans les régions extrasynaptiques.

	\begin{figure}[h]
		\includegraphics[width=\textwidth]{./Images/formation_jnm.png}
		\caption{Formation de la \gls{jnm}.} 
		\descfig{Figure issue de Burden \emph{et al.} 2018 \cite{Burden2018}.}
		\label{fig:FormaJNM}
	\end{figure}
	
	\todo{mettre avec section musk ?}
	Cette accumulation de \glspl{achr} lors du pré-patterning de la jonction, comme d'un certain nombre de protéines synaptiques, est due à un récepteur particulier : \gls{musk}. Ce dernier joue un rôle clef dans la formation de la synapse, sa position déterminant la position de cette dernière \cite{DeChiara1996, Glass1996}.
	
\section{Le récépteur \acrshort{musk}, une molécule clef de la synaptogénèse}
\label{sec:IntroMuSK}	
	\begin{wrapfigure}{l}{0.25\textwidth}
		\centering{\includegraphics[width=0.1\textwidth]{./Images/MuSKReceptor.png}}
		\caption{Récepteur \gls{musk}}
		\descfig{Ig : Domaine immunoglobuline, CRD : \emph{Cysteine Rich Domain}, TM : Domaine transmembranaire,TKD : Domaine tyrosine kinase.}
		\label{fig:RMuSK}
	\end{wrapfigure}
	
	\acrfull{musk} est un récepteur découvert dans l'organe électrique de la raie \emph{Torpedo california} \cite{Jennings1993}. L'expression de ce récepteur à d'abord été mesurée dans les cellules musculaires et localisée au niveau de la \gls{jnm}. \gls{musk} est une récepteur tyrosine-kinase de 98kDa, dans lequel on distingue trois parties : un ectodomaine (partie N-terminale), un domaine transmembranaire, et un domaine cytoplasmique qui porte l'activité kinasique (voir \cref{fig:RMuSK}). 
	
	La partie extracellulaire comporte généralement trois domaines \gls{ig}, dont le domaine \gls{ig}1 a récemment été impliqué dans la liaison avec \gls{lrp}4 \cite{Zhang2011}, ainsi qu'un domaine Frizzled-like, riche en cystéines (\gls{crd}) \cite{Jing2009}.
	
	\gls{musk} possède trois ligands connus : l'Agrine (via \acrshort{lrp}4), un collagène spécifique associé à l'\Gls{ache} appelé \acrshort{colq}, et les \Glspl{wnt}, tous nécessaire au développement complet de la synapse. Un défaut de signalisation de l'un d'entre eux entraîne ainsi des défauts structuraux et/ou fonctionnels de la synapse.
	
	%Précédemment à la fin de section 1.
	L'Agrine est le ligand historique de \gls{musk} \cite{Glass1996}, et est sécrété par l'axone au contact de la cellule musculaire. Plus récemment, des travaux ont montré que l'Agrine se fixait en fait sur le co-récépteur de \gls{musk} : \gls{lrp}4 \cite{Zhang2008,Kim2008}. Suite à l'activation par l'Agrine de \acrshort{lrp}4, deux complexes \gls{musk}/\gls{lrp}4 vont s'assembler, et cet assemblage tétramérique permettrait une phosphorylation optimale de \gls{musk}, et donc une différenciation complète de la synapse et de l'agrégation des \gls{achr} \cite{Zong2012}.
	
	La présence de \gls{musk} dans le cerveau a longtemps été ignorée, du fait de sa faible expression dans cette organe, quantifiée dans le passé par Northern Blot, une méthode de détection des \acrshort{arnm} peu sensible. Cependant, de nouvelles techniques, telle que l'\gls{his} ou bien la \gls{qpcr}, ont permis de montrer que le récepteur était bien présent dans le tissu cérébral, principalement au niveau des neurones du cortex, du cervelet, et de l'hippocampe \cite{Garcia-Osta2006, Ksiazek2007}. Le récepteur \gls{musk} semble aussi être exprimé fortement dans les astrocytes \cite{Sun2016}, à des taux jusqu'à 5 fois supérieur à son expression dans les muscles squelettiques, où avec son co-récepteur \gls{lrp}4 il régulerait la transmission glutamatergiques au travers du relargage d'ATP et une signalisation liée à l'agrine.
	
	Au niveau du \gls{snc}, deux isoformes de \gls{musk} semblent être exprimées \cite{Garcia-Osta2006}. La première isoforme, de 2644pbs, est identique à une isoforme générée par épissage alternatif dans le muscle \cite{Valenzuela1995}, sans qu'aucun rôle ne lui soit connu pour l'instant. La seconde isoforme est plus courte : 2359pbs, et présente une délétion du troisième domaine \gls{ig}. Les deux isoformes présentent une alanine à la position 454 qui remplace une délétion de 8 acides aminé de l'éctodomaine. Une autre isoforme ayant le domaine \gls{ig}3 supprimé serait impliquée dans l'agrégation des \gls{achr} \cite{Hesser1999}.
	
	Grâce à des techniques de knockdown du gène par séquence antisens au niveau de l'hippocampe, il apparaîtrait que la présence de \gls{musk} dans le cerveau serait nécessaire mais non indispensable à la formation de la mémoire à moyen et long-terme \cite{Garcia-Osta2006}. La voie \gls{creb} est une voie impliquée dans la formation de la mémoire au niveau de l'hippocampe \cite{Silva1998, Kandel2012,Kida2014,Ortega-Martinez2015}, qui passerait par la phosphorylation de \gls{creb} suite au relargage de \acrshort{camp}, augmentant son activité transcriptionnelle. Un modèle propose \cite{Garcia-Osta2006} que l'activation de \gls{musk} activerait la cascade de signalisation de \gls{creb}, permettant la consolidation de la mémoire. Ce modèle expliquerait également l'auto-régulation de \gls{musk} \cite{Moore2001}, dont le gène possède dans sa séquence promotrice un élément CRE-like liant \gls{creb} \cite{Kim2005}. De plus, \gls{musk} est nécessaire à la formation de la \gls{ltp} de l'hippocampe \cite{Garcia-Osta2006}.

	Ainsi, le récepteur \gls{musk} possède comme ligand les protéines \glspl{wnt}. Ces protéines, connues pour leurs rôles prépondérant lors de la neurogénèse et de la mise en place des différentes structures du cerveau, semblent avoir un rôle important dans la mise en place de la jonction neuromusculaire, de concert avec \gls{musk}.

\section{Les protéines \Acrshortpl{wnt}, ligands de \acrshort{musk}}
\label{sec:IntroWnt}	
	\begin{wrapfigure}{l}{0.4\textwidth}
		\includegraphics[width=0.4\textwidth]{./Images/WntProtein.png}	
		\caption{Structure d'une protéine \Gls{wnt} classique.}
		\descfig{Figure issue de Willert \& Nusse 2012 \cite{Willert2012}. En orange sont représentés les 22 résidus cystéines.}
		\label{fig:WntProt}
	\end{wrapfigure}
	
	Les \Acrfullpl{wnt} sont des glycoprotéines sécrétées, de 40kDa pour 350 acides aminés, impliquées dans de nombreux processus développementaux tel que l'embryogenèse, la prolifération, la différenciation, la migration cellulaire, ou encore l'apoptose \cite{Miller2002, Willert2012}. De plus, des travaux ont pu montrer que les \Glspl{wnt} étaient également impliquées dans des étapes précoces de la formation de la \gls{jnm} \cite{Hall2000}. En plus de leurs rôles durant le développement, les \Glspl{wnt} jouent également un rôle à l'age adulte dans la maintenance des tissus adultes. 
	
	La structure des \Glspl{wnt} est complexe, avec  de nombreux ponts disulfures caractéristiques de cette famille de protéines, d'hélices \textalpha{}, ainsi que la présence d'un acide palmitoléïque en Ser209 \cite{Takada2006} participant à la liaison avec le récepteur (voir \cref{fig:WntProt}). On peut également observer la présence d'un acide palmitique en position Cys77 \cite{Takada2006} conservé au cours de l'évolution. La présence de ces acides gras rendent les protéines \Gls{wnt} très hydrophobes, ce qui a retardé leurs caractérisations.
	
	On connaît actuellement 19 membres de cette famille de protéine chez la souris et chez l'humain. Classiquement, les \Glspl{wnt} se lient sur le domaine \gls{crd} de leur récepteur canonique \gls{fz}, associé aux co-récepteurs \gls{lrp}5 ou 6, mais il existe d'autres récepteurs non canoniques tels que : \gls{ror} \cite{Cadigan2006, Gordon2006, Green2008}, \gls{ryk} \cite{Bovolenta2006, Fradkin2010}, ou bien encore \gls{musk} \cite{Jing2009}, qui possèdent également un \gls{crd}.
		
	Les protéines \gls{wnt} peuvent activer plusieurs voies de signalisation différentes dans la cellule :  la voie canonique/\textbeta{}-catenin, la voie \gls{pcp}, et d'autres voies indépendante de la \textbeta{}-catenin. Pour la voie canonique, en l'absence de \Glspl{wnt} sur le récepteur \gls{fz}, la \textbeta{}-catenin est continuellement marquée par le complexe \acrshort{gsk3}-\acrshort{apc}-\acrshort{ck1}. Ce complexe va phosphorylé la \textbeta{}-catenin, permettant sa reconnaissance par la \gls{btrcp}, une E3 ubiquitine-ligase, et son marquage pour destruction par le protéasome. Quand les \Glspl{wnt} se lient à \gls{fz}, \emph{Dishevelled} (une protéine centrale dans les différentes voie \glspl{wnt} \cite{Gao2010}) est recrutée à la membrane, permettant l'interaction de \gls{lrp}5/6 avec la \gls{gsk3}. Cette interaction va libéré la \textbeta{}-catenin qui s'accumule dans le cytoplasme, puis se translocalise dans le noyau, d'où elle va avoir un effet sur la transcription des gènes. 
	
	La voie \gls{pcp} est impliquée principalement dans la migration, la polarisation ainsi que le destin cellulaire. Cette voie dans le \gls{snc} est essentielle pour la gastrulation et la formation du tube neural durant l'embryogenèse \cite{Adler2002, Nejsum2009}. L'étude de la voie \gls{pcp} est complexe, car de nombreux acteurs de cette voie sont également impliqués dans d'autres voies de signalisations importantes. Les \Glspl{wnt} en se liant à \gls{fz} vont recrutés \emph{Dishevelled} à la membrane, conduisant à l'activation de \acrshort{rhoa} et \gls{rock}. L'activation de ces protéines va entrainé le remodelage du cytosquelette d'actine et du réseau microtubulaire, nécessaire à la morphogénèse cellulaire. La voie \gls{pcp} active également \acrshort{jnk} et c-JUN activant la transcription de gènes cibles \cite{Niehrs2012}.
	
	Enfin, parmi les voies \glspl{wnt} non canonique, la plus connue est la voie \Gls{wnt}/Calcium, ou \emph{Dishevelled} va activer la \acrshort{plc}, résultant en une augmentation de la concentration intracellulaire d'\acrshort{ip3} et de \acrshort{dag}. L'\acrshort{ip3} provoque l'ouverture des canaux calciques du \gls{re}. Cette augmentation de calcium active la \acrshort{pkc} et la \acrshort{campk2}, qui agissent sur la transcription nucléaires au travers de différents facteurs tels que \acrshort{creb} \cite{Koles2012}. Il est également a noté que dans certains cas, en réponse à une stimulation par \Gls{wnt}3, le récepteur \gls{ryk} était clivé par la \textgamma{}-sécrétase, induisant la translocation de la partie intracellulaire du récepteur dans le noyau \cite{Lyu2008}.
	
	\emph{In vitro}, il a été montrer que plusieurs \Glspl{wnt} interagissaient avec \gls{musk} : \Gls{wnt}2, 3a, 4, 6, 7b, 9a, et 11 \cite{Strochlic2012, Zhang2012, Barik2014}, avec différents effets. Seules \gls{wnt}4, 9a et 11 vont conduire à une dimérisation de \gls{musk} et à son activation (\emph{in vitro}). Ceci est cohérent avec le fait que chez le Poisson-zèbre, l'orthologue de \gls{musk}, \emph{unplugged}, possède aussi un \gls{crd} qui interagit avec des protéines \Glspl{wnt} pour induire l'agrégation de \glspl{achr} \cite{Jing2009, Gordon2012}. \Gls{lrp}4 semble être également nécessaire à l'agrégation des \gls{achr} médié par les \gls{wnt}s \cite{Zhang2012}.

\section{\acrshort{musk} et \Acrshortpl {wnt} : Contexte de l'étude et but du stage}
\label{sec:Contexte}	
	Dans le but d'étudier le rôle de l'interaction des protéines \Glspl{wnt} et du domaine \gls{crd} de \gls{musk}, l'équipe de C. LEGAY à crée une souris transgénique dont le \gls{crd} était supprimé (\mcrd) \cite{Messeant2015, Messeant2017}. Il a ainsi été montré que le \gls{crd} était nécessaire à la \gls{jnm} à la fois pour sa formation et pour son maintien à l'age adulte, et que \Gls{wnt}4 et 11 participaient activement à la formation de cette dernière. De plus, un traitement au \gls{licl} (inhibiteur de la \gls{gsk3}) permettait à la \gls{jnm} un retour vers un phénotype sauvage. 
	
	En plus de leurs problèmes musculaires, les souris \mcrd exhibaient des défauts centraux : durant son stage, une étudiante, Bertille SOMON, a montré que les mutants mâles avaient des blessures importantes au niveau du dos, blessures qui n'étaient pas dues à des comportements d'agressivité entre souris. De plus, une analyse comportementale a été réalisée en collaboration avec le groupe du Dr LANFUMEY (Centre de Psychiatrie et Neurosciences, Paris), et le test \gls{nor} a révélé que les souris mutantes souffraient d'un déficit de la mémoire intermédiaire.
	
	Comme \gls{musk} est exprimé dans le cerveau adulte, principalement au niveau de l'hippocampe \cite{Garcia-Osta2006}, et que ce lieu joue un rôle prépondérant dans la formation de la mémoire intermédiaire, l'objectif de mon stage va être d'explorer le rôle de l'interaction de \gls{musk} et des \Glspl{wnt} dans le cerveau, utilisant pour cela les souris \mcrd. Je poserai au cours de mon stages plusieurs questions : Quelle est l'origine des blessures observées chez le mâle, est-ce que la structure du cerveau est affectées chez le mutant, quelles sont les cellules exprimant \gls{musk}, et quel est le niveau d'expression du gène dans différentes structures du cerveau.

%Cela se fera au travers de 5 axes : 
%\begin{enumerate}
%	\item \sout{Quelle est l'origine des blessures chez le mâle ?}
%	\item La structure du cerveau est-elle affectée chez le mutant ?
%	\item Quelles sont les cellules exprimant \gls{musk} ?
%	\item Quel est le niveau d'expression de \gls{musk}/\mcrd dans le cerveau ?
%	\item \sout{Un traitement au \gls{licl} peut-il permettre un retour du mutant à un phénotype sauvage au niveau du comportement ?}
%\end{enumerate}