\Acrshort{musk} est un récepteur tyrosine kinase connut principalement pour son rôle crucial à la \acrshort{jnm}. Ce récepteur va permettre la mise en place de l'élément post-synaptique, le muscle, avant l'arrivée de l'axone, ainsi que la maturation de la synapse par la suite. Le récepteur \acrshort{musk} possède également un domaine \acrshort{crd} de liaison aux protéines \acrshort{wnt}, qui participe à la mise en place de la \acrshort{jnm}. Cependant, le rôle de \acrshort{musk} et du \acrshort{crd} dans le cerveau reste méconnu. Au travers des techniques de coloration histologique, d'immunomarquage et de \acrshort{qpcr}, je montre que que le récepteur \gls{musk} est localisé dans des endroits discret du cerveau et est exprimé par les astrocytes. De plus, le récepteur va être exprimé de manière différentes par l'hippocampe gauche ou droit. Enfin, je montre que la délétion du domaine \acrshort{crd} du récepteur va avoir pour effet de modifier la distribution des neurones dans différentes régions de l'hippocampe.