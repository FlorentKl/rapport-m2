\Acrshort{musk} est un récepteur tyrosine kinase connu pour son rôle crucial dans la formation et la maintenance de la jonction neuromusculaire. Ce récepteur est activé par 3 ligands : l’agrine, \acrshort{colq}  et les \acrshortpl{wnt}. Ces dernières se lient sur un domaine \acrshort{crd} dans l’ectodomaine de \acrshort{musk} nécessaire à la synaptogenèse. Cependant, le rôle de MuSK et du domaine CRD dans le cerveau reste méconnu. L'utilisation de techniques de coloration histologique, d'immunomarquage et de \acrshort{qpcr} m'ont permis de montrer que le récepteur \acrshort{musk} est localisé dans des endroits discrets du cerveau (Hippocampe notamment) et est exprimé par des astrocytes de type fibreux et des neurones. De plus, le \acrshort{crd} participe à la structuration des couches pyramidale et granulaire de l'hippocampe. Enfin, je montre que le récepteur est exprimé différemment par l'hippocampe gauche ou droit. Ce travail est le premier à s'intéresser spécifiquement à la localisation cellulaire de la protéine \acrshort{musk} et représente un premier pas dans l'étude du rôle comportementale du récepteur.