%%% Langue
\usepackage{polyglossia} %Langue
\setmainlanguage{french}
\setotherlanguage{english}

%%% Aspect Documents
\usepackage{geometry}
\geometry{a4paper,
			top=2.5cm,
			bottom=2.5cm,
			left=2.5cm,
			right=2.5cm}

%%% Police / Présentation
\usepackage{fontspec} %Police
\setmainfont[Ligatures=TeX,
			Kerning=Uppercase, %Meilleurs intégration Maj-Min
			BoldFont={AGaramondPro-Bold},
			ItalicFont={AGaramondPro-Italic},
			Numbers=Proportional,]{Adobe Garamond Pro} %test
\usepackage{microtype}
\usepackage{unicode-math} %Plante si absent mais formule dans texte
\setmathrm{Adobe Garamond Pro}
%\usepackage[adobe-garamond]{mathdesign} %Police signe maths %%Fait buguer avec unicode-math
\usepackage[version=4]{mhchem} %Ecriture ion/molécules
\usepackage{textgreek}

%%% Présentation
\usepackage{csquotes}
\usepackage{parskip}
\setlength{\parindent}{3em}
\usepackage{caption} %Titre figure mieux géré
\captionsetup{justification=raggedright, 
			singlelinecheck=false}
\usepackage{subcaption} %idem sous figures
\usepackage{titlesec}
\usepackage{titling} %Utilisation titre dans en-tête
\usepackage{xspace} %Gestion espace dans macro
\usepackage{fancyhdr} %Gestion en-tête et pieds de page
\fancypagestyle{plain}{ %Redef style page plain de page avec Chapitre
	\fancyhf{} % clear all header and footer fields
	\fancyhead[C]{\textit{\titredoc}}
	\fancyfoot[C]{\textbf{\thepage}} % except the center
	\renewcommand{\headrulewidth}{0,4pt}
	\renewcommand{\footrulewidth}{0pt}}
\setlength{\headsep}{25pt}

%%% Images
\usepackage{graphicx} %Gestion des images
\usepackage{wrapfig} %Gestion des images, permet mise image et texte cote a cote
\captionsetup[figure]{labelfont={bf},labelformat={default},labelsep=period,name={Figure }} %Affichage Figure x. au lieu de fig x.

%%% Pdf / Références internes
\usepackage{hyperref} %Meilleure gestion pdf, load avant glossaries pour crossref
\hypersetup{pdfauthor={Florent KLEE},
	pdftitle={Roles de Wnts et MuSK, un récepteur tyrosine kinase dans le cerveau},
	pdfsubject={Roles de Wnts et MuSK dans le cerveau},
	colorlinks,
	citecolor=black,
	filecolor=black,
	linkcolor=black,
	urlcolor=blue,}
\usepackage{fancyref} %Après Hyperef
\usepackage{cleveref} %Ref mieux gérer

%%% Glossaire
\usepackage[toc, %Apparaît dans table des matières
			nopostdot, %Pas de points a fin acronyms
			nonumberlist,%Pas de numéro de pages
			nogroupskip, %Pas de regroupement par première lettre
			]{glossaries} 
							
%%% Biblio
\usepackage[sorting=none,
			backend=biber,
			style=FKStyle,
			block=none,
			date=year,
			minbibnames=3, maxbibnames=9,
			]{biblatex} %A voir + tard
\addbibresource{library.bib}
\defbibheading{subbibliography}[\refname]{\section*{#1}}
\usepackage[nottoc]{tocbibind} %Bibliographie dans Table des matières

%%% Divers
\usepackage{todonotes}

%%% Création Index
\makeindex

%Redef format chapitre, notamment reduire espace au dessus et dessous de ceux ci
\makeatletter
\titleformat{\chapter}[hang]
{\normalfont\huge\bfseries}{\chaptertitlename~\thechapter : }{17\p@}{\huge}
\titlespacing*{\chapter}{0pt}{10\p@}{13\p@}
\makeatother

%%%%% COMMANDES / MACRO %%%%%
\newcommand{\blankpage}{\newpage\thispagestyle{empty}\null\newpage} %Création page blanche
\newcommand{\up}[1]{\textsuperscript{#1}} %Permet exposants
\newcommand{\mcrd}{MuSK\textDelta{}CRD\xspace} %Ecrire MuSK-CRD
\newcommand{\titredoc}{Rôles de Wnts et MuSK,\\un récepteur tyrosine kinase dans le cerveau }
\renewcommand{\thefigure}{\arabic{figure}} %Permet "Figure X" au lieu de "Figure Y.X" (Y étant num chapitre, X num figure)