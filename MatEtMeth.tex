\chapter{Matériel et Méthodes}

\section{Animaux et Prélèvements}
\label{sec:AnimEtPrelev}
	Les souris utilisées sont issues d'une lignée hétérozygote issue d'une fond génétique C57BL/6 mixte \cite{Messeant2015, Messeant2017}. Les souris mutantes \mcrd et \gls{wt} testées sont issues de même portée. 
	Afin de prélever les cerveaux, les souris ont été euthanasiées à l'aide d'une injection péritonéale de Pentorbital. Une injection péristaltique de \gls{pfa} 4\% est réalisée pour fixé les tissus. Le cerveau est retiré et post-fixé dans du \gls{pfa} 4\% pendant 1 heure, puis transféré dans une solution de sucrose 30\% à 4°C jusqu'à utilisation.
	
\section{\acrshort{qpcr}}
\label{sec:qPCR}
	Les hippocampes droits et gauches sont prélevés du cerveau et immédiatement plongés dans 800µl de TRIzol\textregistered, broyés mécaniquement puis stockés à -20°C. L'extraction de l'\gls{arn} se fait grâce au RNeasy Protect Mini Kit de Qiagen\textregistered. Le traitement DNAse, puis la \gls{rtpcr}, se font grâce au RT\up{2} First Strand Kit de Qiagen\textregistered. La \gls{qpcr} se réalise avec SYBR Green/ROX qPCR MM (ThermoFisher\textregistered). Les primers utilisés sont dirigés contre \gls{musk} et \acrshort{26s}.
	
\section{Marquages}
\label{sec:Marquages}
	Les cerveaux sont sectionnés à une épaisseur de 50µm au microtome. Une coloration de Nissl est réalisée sur une coupe sur 3, montée sur lame avec de la gélatine. Après séchage à l'air libre pendant 24 heures, les lames vont être déshydratées et dégraissées avec des bains successifs dans de l'éthanol 70°, 95°, 100° et de xylène. Les lames vont être ensuite réhydratée avec des bains d'éthanol de concentration décroissante avant d'être plongées dans le colorant de Nissl (solution de Crésyl Violet).  Les lames sont ensuite  à nouveau déshydratées et immergées dans un bain de xylène. Le montage se fera dans du milieu Eukitt\textregistered.
	
	Pour l'immunomarquage, une coupe sur 6 centrée sur l'hippocampe est choisie. Les coupes flottantes sont incubés 1 heure à températures ambiante dans un sérum de blocage (\gls{pbs}, Sérum de chèvre 5\%, \gls{bsa} 3\%, Triton 0.2\%), puis sont incubés dans 500µl de solutions de blocage avec anticorps pendant une nuit à 4°C avec agitation. Le lendemain, les coupes sont rinçées 3 fois 7 minutes dans une solution de rinçage/blocage (\gls{pbs}, \gls{bsa} 2,5µg/µl et Triton 1,5\%), puis incubés 1h45 avec l'anticorps secondaire dans le noir à température ambiante, puis 10 minutes avec du \gls{dapi}, sous agitation. Les lames sont à nouveau rincées 3 fois 7 minutes dans la solution précédente. Les lames sont montés dans du milieu de montage Mowiol ou FluoroMount-G.
	
	Les anticorps primaires utilisés sont les suivants : Anti-\acrshort{musk} (polyclonal de lapin, ABS549, Millipore-Merck),  Anti-\acrshort{neun} (polyclonal de lapin, ab104225, Abcam), Anti-\acrshort{gfap} (polyclonal de souris, MAB360, Millipore-Merck). Pour les anticorps secondaires, les suivants ont été utilisés : Anti-rabbit-AlexaFluor\textregistered488 (polyclonal de chèvre, A11008, InVitroGen), Anti-mouse-Cy3 (polyclonal de chèvre, 115-165-003, Jackson ImmunoResearch).
	
\section{Acquisition d'images et Statistiques}
\label{sec:ImagesStats}
	Les images ont été pris avec une microscope Epifluo Nikon Eclypse TE-2000E équipé d'une caméra Photometrics CoolSNAP HQ2 (Fluorescence), et d'une caméra Qimaging Color Retiga 2000R (Lumière blanche).
	Les données ont été mesurées sur images grâce au logiciel ImageJ \cite{Schneider2012}.