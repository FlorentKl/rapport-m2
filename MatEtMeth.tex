\chapter{Matériel et Méthodes}

\section{Animaux et Prélèvements}
\label{sec:AnimEtPrelev}
	Les souris utilisées sont issues d'une lignée hétérozygote venant d'un fond génétique mixte \cite{Messeant2015, Messeant2017}. Les souris mutantes \mcrd et \gls{wt} testées sont issues de même portée. Afin de prélever les cerveaux, les souris ont été euthanasiées à l'aide d'une injection péritonéale de Pentorbital. Une injection péristaltique de \gls{pfa} 4\% est réalisée pour fixer les tissus. Le cerveau est prélevé et post-fixé dans du \gls{pfa} 4\% pendant 1 heure, puis transféré dans une solution de sucrose 30\% à 4°C jusqu'à utilisation.
	
\section{\acrshort{qpcr}}
\label{sec:qPCR}
	Les hippocampes droit et gauche sont prélevés du cerveau et immédiatement plongés dans 800µl de TRIzol\textregistered, broyés mécaniquement puis stockés à -20°C. L'extraction de l'\acrshort{arn} se fait grâce au RNeasy Protect Mini Kit de Qiagen\textregistered. Le traitement DNAse, puis la \gls{rtpcr}, sont réalisées avec RT\up{2} First Strand Kit de Qiagen\textregistered. La \gls{qpcr} se réalise avec SYBR Green/ROX qPCR MM (ThermoFisher\textregistered). Les primers utilisés sont dirigés contre \gls{musk} et \acrshort{26s}.
	
\section{Marquages}
\label{sec:Marquages}
	Les cerveaux sont sectionnés à une épaisseur de 50µm au microtome puis les coupes sont montées sur lame avec de la gélatine. Après séchage à l'air libre pendant 24 heures, les lames vont être déshydratées et dégraissées dans des bains successifs d'éthanol 70°, 95°, 100° et de xylène. Les lames vont être ensuite réhydratées dans des bains d'éthanol de concentration décroissante avant d'être plongées dans le colorant de Nissl (solution de Crésyl Violet).  Les lames sont ensuite  à nouveau déshydratées et immergées dans un bain de xylène. Le montage se fera dans du milieu Eukitt\textregistered.
	
	Pour l'immunomarquage, des coupes de 40µm sont réalisées au cryostat, et récupérés dans des puits avec du \acrshort{pbs} 0,1M. Lors du marquages, les coupes sont incubées 1 heure à température ambiante dans un sérum de blocage (\acrshort{pbs}, Sérum de chèvre 5\%, \acrshort{bsa} 3\%, Triton 0.2\%), puis  incubées dans 500µl de solution de blocage avec les anticorps pendant une nuit à 4°C avec agitation. Le lendemain, les coupes sont rinçées dans la solution de blocage diluée, incubées 1h45 avec l'anticorps secondaire dans le noir à température ambiante, puis 10 minutes avec du \gls{dapi}, sous agitation. Les lames sont à nouveau rinçées dans la solution précédente, et sont montées dans du milieu de montage Mowiol ou FluoroMount-G.
	
\begin{table}
		\begin{tabular}{l l l}
		\hline
		Anticorps 					    & Source 					& Dilution \\
		\hline
		\acrshort{musk} 	    & Millipore 				& 1:500 \\
		\acrshort{neun} 	     & Abcam 					& 1:500 \\
		\acrshort{gfap} 	      & Abcam 					 & 1:100 \\
		Anti-Rabbit-A488      & Invitrogen 				& 1:500\\
		Anti-Mouse-A594 	& Invitrogen 			  & 1:500\\
		\hline
	\end{tabular}
\caption{Anticorps utilisés}
\label{TableAc}
\end{table}
	
	Les anticorps primaires utilisés sont les suivants : Anti-\acrshort{musk} (polyclonal de lapin, ABS549, Millipore-Merck),  Anti-\acrshort{neun} (polyclonal de lapin, ab104225, Abcam), Anti-\acrshort{gfap} (polyclonal de souris, MAB360, Millipore-Merck). Pour les anticorps secondaires, les suivants ont été utilisés : Anti-rabbit-AlexaFluor\textregistered488 (polyclonal de chèvre, A11008, InVitroGen), Anti-mouse-Cy3 (polyclonal de chèvre, 115-165-003, Jackson ImmunoResearch).
	
\section{Acquisition d'images et Statistiques}
\label{sec:ImagesStats}
	Les images ont été prises avec un microscope Epifluo Nikon Eclypse TE-2000E équipé d'une caméra Photometrics CoolSNAP HQ2 (Fluorescence) et d'une caméra Qimaging Color Retiga 2000R (Lumière blanche).
	Les images ont été traitées sur le logiciel ImageJ \cite{Schneider2012}.