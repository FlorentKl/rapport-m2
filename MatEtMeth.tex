\paragraph{Animaux et Prélèvements}
\label{sec:AnimEtPrelev}
	Les souris utilisées sont issues d'une lignée hétérozygote provenant d'un fond génétique mixte \cite{Messeant2015, Messeant2017}. Les souris mutantes \mcrd et \gls{wt} testées sont issues de même portée. Afin de prélever les cerveaux, les souris sont  euthanasiées à l'aide d'une injection intrapéritonéale de Pentobarbital (Ceva\textregistered) à une dose de 40µg/mg. Une perfusion péristaltique de \gls{pfa} 4\% est réalisée pour fixer les tissus. Le cerveau est prélevé et post-fixé dans du \gls{pfa} 4\% pendant 1 heure, puis transféré dans une solution de sucrose 30\% à 4°C jusqu'à utilisation.
	
\paragraph{Génotypage}
\label{sec:genotypage}
	Les animaux sont génotypés à l'âge de 7 jours par prélèvement du pouce d'une patte avant. L'extraction d'\acrshort{adn} se fait par lyse alcaline à 95°C pendant 20 minutes puis ajout de Tris-HCl 40mM. Une \gls{pcr} est réalisée avec les couples de primers Ef et Lxr  pour les individus ??? \todo{wt ou mut ?}, et Ef et Wr pour les individus ???\todo{idem ?} (voir \cref{table:génotypage} pour la séquence des primers utilisés). Dépôt des produits de \gls{pcr} dans un gel d'agarose 2\% et migration à 200V pendant 10 minutes. Révélations des bandes au ultraviolet après coloration au \acrshort{bet}.
	
	\begin{table}[h]
		\centering{
			\begin{tabular}{|c || c|}
				\hline
				Primers 				& Séquence \\
				\hline
				Ef 								& 5'-CTCTTCTCCCTTCTGCCCACCGAT-3' \\
				\hline
				Lxr								& 5'-AGTTATACTAGAGCGGCCGTTCACCG-3'\\
				\hline
				Wr								& 5'-CCCTGGGAATATGGTTTCTCATTGCT-3' \\
				\hline	
			\end{tabular}
		}
		\caption{Séquence des primers utilisés pour le génotypage}
		\label{table:génotypage}
	\end{table}
\FloatBarrier
	
\paragraph{Real-Time \acrshort{rtpcr} (\acrshort{qpcr})}
\label{sec:qPCR}
	Les hippocampes droits et gauches ont été disséqués et immédiatement plongés dans 800µl de TRIzol\textregistered, broyés mécaniquement puis stockés à -20°C. L'extraction de l'\acrshort{arn} a été faite grâce au RNeasy Protect Mini Kit (Qiagen\textregistered). Le traitement à la DNAse, puis la transcription inverse de ARN suivie d'une PCR (\acrshort{rtpcr}), ont été réalisées avec RT\up{2} First Strand Kit (Qiagen\textregistered). La \gls{qpcr} est faite avec du SYBR Green/ROX qPCR Master Mix (ThermoFisher\textregistered) contenant une Taq Polymerase et des \glspl{dntp}. Les primers utilisés sont dirigés contre \gls{musk} et \acrshort{26s} (gène de ménage).
	
\paragraph{Coloration de Nissl}
\label{sec:Nissl}
	Des coupes de cerveaux de 50µm sont réalisées au microtome puis montées sur lame avec de la gélatine. Après séchage à l'air libre pendant 24 heures, les lames sont déshydratées et dégraissées dans des bains successifs d'éthanol 70°, 95°, 100° et de xylène. Les lames sont ensuite réhydratées dans des bains d'éthanol de concentrations décroissantes avant d'être plongées dans le colorant de Nissl (solution de Crésyl Violet).  Les lames sont ensuite  à nouveau déshydratées et immergées dans un bain de xylène. Le montage des coupes est fait dans du milieu Eukitt\textregistered.
	
\paragraph{Immunomarquage}	
\label{sec:immunomarquage}
	Des coupes de 40µm sont réalisées au cryostat, et récupérés dans des puits avec du \acrshort{pbs} 0.1M. Lors du marquages, les coupes sont incubées 1 heure à température ambiante dans un sérum de blocage (\acrshort{pbs}, Sérum de chèvre 5\%, \acrshort{bsa} 3\%, Triton X-100 0.2\%), puis  incubées dans 500µl de solution de blocage avec les anticorps pendant une nuit à 4°C avec agitation. Le lendemain, les coupes sont rinçées dans la solution de blocage diluée, incubées 2 heures avec l'anticorps secondaire dans le noir à température ambiante, puis 20 secondes avec du \gls{dapi}. Les lames sont à nouveau rinçées dans la solution précédente, et sont montées dans du milieu de montage Mowiol ou FluoroMount-G\textregistered.
	Les anticorps utilisés sont présentés dans la \cref{table:Ac}.

	\begin{table}[h]
		\centering{
			\begin{tabular}{l l l l}
			\hline		
			\teteCol{Anticorps} 	& \teteCol{Source} 	&\teteCol{Réf.}	& \teteCol{Dilution}\\
			\hline 
			\acrshort{musk} 	    & Millipore 				&ABS549 	        & 1:500 \\
			\acrshort{neun} 	     & Abcam 					&ab105225        & 1:500 \\
			\acrshort{gfap} 	      & Abcam 					 &ab4648			  & 1:50 \\
			Anti-Rabbit-A488      & Invitrogen 				&A11008				& 1:500\\
			Anti-Mouse-A594 	& Invitrogen 			  &A11005			  & 1:500\\
			\hline
		\end{tabular}
		\caption{Références des Anticorps et dilution utilisée.}
		\label{table:Ac}
		}
	\end{table} 
\FloatBarrier
	
\paragraph{Acquisition d'images et Statistiques}
\label{sec:ImagesStats}
	Les images ont été prises avec un microscope Epifluo Nikon Eclypse TE-2000E équipé d'une caméra Photometrics CoolSNAP HQ2 (Fluorescence) et d'une caméra Qimaging Color Retiga 2000R (Lumière blanche).
	Les images ont été traitées sur le logiciel ImageJ \cite{Schneider2012}.