 % !tex root= main.tex
\paragraph{Animaux et Prélèvements}
\label{par:AnimEtPrelev}
	Les souris utilisées sont issues d'une lignée hétérozygote provenant d'un fond génétique mixte \cite{Messeant2015, Messeant2017}. Les souris mutantes \mcrd et \gls{wt} testées sont issues de même portée. Afin de prélever les cerveaux, les souris sont  euthanasiées à l'aide d'une injection intrapéritonéale de Pentobarbital (Ceva\textregistered) à une dose de 40µg/mg. Une perfusion péristaltique de \gls{pfa} 4\% est réalisée pour fixer les tissus. Le cerveau est prélevé et post-fixé dans du \gls{pfa} 4\% pendant 1 heure, puis transféré dans une solution de sucrose 30\% à 4°C jusqu'à utilisation. 
	
\paragraph{Génotypage}
\label{par:genotypage}
	Les animaux sont génotypés à l'âge de 7 jours par prélèvement du pouce d'une patte avant. L'extraction d'\acrshort{adn} se fait par lyse alcaline à 95°C pendant 20 minutes puis ajout de Tris-HCl 40mM. Une \gls{pcr} est réalisée avec les couples de primers Ef et Lxr  pour les individus ??? \todo{wt ou mut ?}, et Ef et Wr pour les individus ???\todo{idem ?} (voir \cref{table:génotypage} pour la séquence des primers utilisés). Dépôt des produits de \gls{pcr} dans un gel d'agarose 2\% et migration à 200V pendant 10 minutes. Révélations des bandes au ultraviolet après coloration au \acrshort{bet}.
	
	\begin{table}[h]
		\centering{
			\caption{Séquence des primers utilisés pour le génotypage}
			\begin{tabular}{|c || c|}
				\hline
				Primers 						& Séquence \\
				\hline
				Ef 								& 5'-CTC TTC TCC CTT CTG CCC ACC GAT-3' \\
				\hline
				Lxr								& 5'-AGT TAT ACT AGA GCG GCC GTT CAC CG-3'\\
				\hline
				Wr								& 5'-CCC TGG GAA TAT GGT TTC TCA TTG CT-3' \\
				\hline	
			\end{tabular}
		}
		\label{table:génotypage}
	\end{table}
\FloatBarrier
	
\paragraph{Real-Time \acrshort{rtpcr} (\acrshort{qpcr})}
\label{par:qPCR}
	Les hippocampes droits et gauches ont été disséqués et immédiatement plongés dans 800µl de TRIzol\textregistered, broyés mécaniquement avec un TissuRuptor II (Qiagen\textregistered) puis stockés à -20°C. L'extraction de l'\acrshort{arn} a été faite grâce au RNeasy Protect Mini Kit (Qiagen\textregistered). Le traitement à la DNAse, puis la transcription inverse de ARN suivie d'une PCR (\acrshort{rtpcr}), ont été réalisées avec RT\up{2} First Strand Kit (Qiagen\textregistered). La \gls{qpcr} est faite avec du SYBR Green/ROX qPCR Master Mix (ThermoFisher\textregistered) contenant une Taq Polymerase et des \glspl{dntp}. Les primers utilisés sont dirigés contre \gls{musk} et \acrshort{26s} (gène de ménage).
	
\paragraph{Coloration de Nissl}
\label{par:Nissl}
	Des coupes de cerveaux de 50µm sont réalisées au microtome à partir du cerveau de 4 souris : 1 femelle mutante, 1 mâle mutant, 1 femelle sauvage et 1 mâle sauvage. Une coupe sur 3 est séléctionnée puis montée sur lame avec de la gélatine. Après séchage à l'air libre pendant 24 heures, les lames sont déshydratées et dégraissées dans des bains successifs d'éthanol 70°, 95°, 100° et de xylène. Les lames sont ensuite réhydratées dans des bains d'éthanol de concentrations décroissantes puis d'eau déminéralisée avant d'être plongées dans le colorant de Nissl (solution de Crésyl Violet).  Les lames sont rinçées dans de l'eau déminéralisée puis à nouveau déshydratées et immergées dans un bain de xylène. Le montage des coupes est fait dans du milieu Eukitt\textregistered (Sigma).
	
\paragraph{\Acrlong{ihc}}	
\label{par:ihc}
	Pour l'\Gls{ihc}, des coupes de 40µm sont réalisées au cryostat, et récupérés dans des puits avec du \acrshort{pbs} 0.1M. Lors du marquages, les coupes sont incubées 1 heure à température ambiante dans un sérum de blocage (\acrshort{pbs}, Sérum de chèvre 5\%, \acrshort{bsa} 3\%, Triton X-100 0.2\%), puis  incubées dans 500µl de solution de blocage avec les anticorps pendant une nuit à 4°C avec agitation. Le lendemain, les coupes sont rinçées dans la solution de blocage diluée, incubées 2 heures avec l'anticorps secondaire dans le noir à température ambiante, puis 20 secondes avec du \gls{dapi}. Les lames sont à nouveau rinçées dans la solution précédente, et sont montées dans du milieu de montage Mowiol ou FluoroMount-G\textregistered.
	
	\paragraph{Lyse des tissus}
	\label{par:lyse}
	Les tissus disséqués sont immédiatement congelés par immersion dans de l'azote liquide, puis stockés à -80°C. Ceux-ci sont ensuite rassemblés avant d'être plongés dans du tampon RIPA (10mL de tampon par gramme d'échantillons) additionné d'un cocktail d'inhibiteur de protéases (1:25). Les échantillons sont ensuite broyés mécaniquement avec un TissuRuptor II (Qiagen\textregistered) puis incubés sur de la glace pendant 1 heure avec agitation. Après une centrifugation à 20.000G, 4°C pendant 15min, le surnageant est récupéré et stockés à -20°C. 
	
	\paragraph{\Acrlong{ip}}
	\label{par:ip}
	Le dosage des protéines est effectué par BC Assay. L'équivalent de 3mg de protéines est dilué dans 500µL de tampon RIPA auquel est ajouté 5µL d'anticorps (non dilué), puis incubé sur la nuit, 4°C avec mélange. Des billes magnétiques (Dynabead\textregistered Protein G) sont lavées puis mise en incubation avec la solution antigène-anticorps 3 heures à 4°C avec mélange. Le surnageant est récupéré et mis de coté comme contrôle. Les billes sont lavées dans du tampon RIPA, puis avec du tampon Tris 50mM. La séparation des protéines et des billes est réalisées en ajoutant du Bleu de Laemmli 1X chauffés à 70°C 10 minutes. Le surnageant est récupéré et stocké à -20°C. 
	
	\paragraph{\Acrlong{wb}}
	\label{par:wb}
	Les échantillons sont migrés sur un gel Mini-PROTEAN\textregistered{}  TGX 10\% à 80V pendant 10 minutes puis 120V dans du tampon Tris/Glycine/SDS pour ensuite être transférés sur une membrane de nitrocellulose sur la nuit à 4°C, 40V, avec du tampon Tris/Glycine. La membrane est ensuite colorée au Rouge Ponceau afin de vérifier la présence de protéines, décoloré par lavage dans du \acrshort{pbst}, mise 2 heures avec agitation dans du tampon de blocage (Pierce™ Protein-Free (PBS) Blocking Buffer) puis incubés avec l'anticorps primaire dilué dans le tampon de blocage une nuit à 4°C avec agitation. La membrane est lavée dans du \acrshort{pbst} puis incubée avec l'anticorps secondaire (dilué dans du tampon de blocage) 45 minutes. La révélation est faite par un kit Amersham\texttrademark ECL\texttrademark prime.
		
	\paragraph{Anticorps}
	\label{par:anticorps}
	Les anticorps utilisés pour l'\gls{ihc}{} et l'\gls{ip}{} sont présentés dans la \cref{table:Ac}.\todo{Ajouter ref anticorps secondaire}
	
	\begin{table}[h]
		\centering{
		 	\caption{Références des Anticorps et dilution utilisée.}
			\begin{tabular}{c l l l l}
			\hline		
				& \teteCol{Anticorps} 	& \teteCol{Source} 	&\teteCol{Réf.}	& \teteCol{Dilution}\\
			\hline 
			%%% Immunomarquage
				\Acrshort{ihc}	&Rabbit Anti-\acrshort{musk}	&Millipore		&ABS549		&1:500 \\
								&Rabbit anti-\acrshort{neun}	&Abcam			&ab105225	&1:500 \\
								&Mouse anti-\acrshort{gfap}		&Millipore 			&MAB360		&1:1000 \\
								&Goat Anti-Rabbit-A488			&Invitrogen		&A11008		&1:500\\
								&Goat Anti-Mouse-A594			&Invitrogen 	&A11005		&1:500\\
			\hline
			%%% Immunoprécipitation
			\Acrshort{ip} 		& \acrshort{musk} 				&Millipore 		&ABS549		&1:1 000\\
											&Anti-Rabbit-\acrshort{hrp} 		&Amersham	&NA934	&1:20 000\\
			\hline
		\end{tabular}
		\label{table:Ac}
		}
	\end{table} 
\FloatBarrier
	
\paragraph{Acquisition d'images et Statistiques}
\label{par:ImagesStats}
	Les images ont été prises avec un microscope Epifluo Nikon Eclypse TE-2000E équipé d'une caméra Photometrics CoolSNAP HQ2 (Fluorescence) et d'une caméra Qimaging Color Retiga 2000R (Lumière blanche).
	Les images ont été traitées sur le logiciel ImageJ \cite{Schneider2012}.