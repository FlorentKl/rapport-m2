\section{Immunoprécipitation}
\label{sec:IPresultat}
Afin de confirmer la détection de \gls{musk} dans diverse région du cerveau, une immunoprécipitation suivie d'un \gls{wb} a été réalisée sur l'hippocampe, le cervelet et le cortex de trois souris C57Bl/6 sauvage. Un \gls{wb} de \gls{musk} a déjà été décrit sur des diverses structures du cerveau \cite{Garcia-Osta2006}, mais à partir d'extrait totale. Durant son stage, B. SOMON a tentée de réaliser un \gls{ip} suivi d'un \gls{wb}, mais sans obtenir de résultats probant. Ici, la principale modification apportée au protocole est l'utilisation de tampon RIPA (adjonction de déoxycholate de sodium et de \acrshort{sds}) lors de l'extraction des protéines.

\begin{wrapfigure}{l}{0.45\textwidth}
	\missingfigure{WB musk : c'est joli, hein ?}
	\caption{WB après IP}{On peut observer ici...rien du tout !}
	\label{fig:WBMuSK}
\end{wrapfigure}