\section{Structure du cerveau}
\label{sec:NisslResultat}

	\subsection{Organisation globale du cerveau}
	\label{ssec:orgaglobale}
		La coloration de Nissl, ici à base de Crésyl violet, est un marquage classique du tissu nerveux avec une molécule basique, qui marque fortement l'acide nucléique (\acrshort{adn} et \acrshort{arn}) des cellules car basophile (ces structures au microscope prennent le nom de "Corps de Nissl'). Ce marquage est important dans les neurones, riches en \gls{re} rugueux, organite associé à une grande quantité d'\acrshort{arn}. Cette coloration permet ainsi de visualiser les différentes structures cellulaires au sein du tissu nerveux.
	
		Comme les protéines \gls{wnt} sont généralement impliquées dans le développement et l'organisation du cerveau, on peut alors se demander si la suppression du domaine de liaisons de ces protéines sur le récepteur \gls{musk} allait entraîner des modifications du développement du système nerveux qui serait visible avec une coloration histologique.
	
		Afin de comparer l'organisation structurale du cerveau chez la souris sauvage et mutante, une coloration de Nissl a donc été réalisée sur 4 individus : 2 souris mutantes et 2 souris sauvages (1 mâle et 1 femelle pour chaque groupe) (\cref{fig:NisslResultat}). Ce marquage a permis de mettre en évidence que la mutation de \gls{musk} ne modifiait pas l'organisation globale du cerveau, à la fois chez les souris femelles (\cref{fig:FemWTNissl,fig:FemMutNissl}) et mâles (\cref{fig:MaleWTNissl,fig:MaleMutNissl}). 
	
		\begin{figure}[h] %Figure Nissl Résultats
			\begin{center}
				\begin{subfigure}[h]{0.49\textwidth}%F437 WT Nissl 33 l2.tif
					\caption{}
					\label{fig:FemWTNissl}
					\includegraphics[width=\textwidth]{./Images/Nissl/FemWT.jpg}
				\end{subfigure}
				\begin{subfigure}[h]{0.49\textwidth}%F435 Mut Nissl 38.tif
					\caption{}
					\label{fig:FemMutNissl}
					\includegraphics[width=\textwidth]{./Images/Nissl/FemMut.jpg}
				\end{subfigure}
				\begin{subfigure}[h]{0.49\textwidth}%M2 WT Nissl #031.tif
					\caption{}
					\label{fig:MaleWTNissl}
					\includegraphics[width=\textwidth]{./Images/Nissl/MaleWT.jpg}
				\end{subfigure}
				\begin{subfigure}[h]{0.49\textwidth}%M442 Mut Nissl lame 3 36.tif
					\caption{}
					\label{fig:MaleMutNissl}
					\includegraphics[width=\textwidth]{./Images/Nissl/MaleMut.jpg}
				\end{subfigure}
			\end{center}
			\caption{Les souris \mcrd ne présentent pas de défauts de structure globale du \acrshort{snc}.}
			\descfig{%
				Coloration de Nissl sur : 
				\subref{fig:FemWTNissl} Femelle sauvage, 
				\subref{fig:FemMutNissl} Femelle mutante, 
				\subref{fig:MaleWTNissl} Mâle sauvage, 
				\subref{fig:MaleMutNissl} Mâle mutant. 
				Images représentatives de coupes réalisées au niveau de l'hippocampe. 
				Âge moyen des souris : 30 jours. 
				Barre d'échelle : 2 mm.
				n = 4 (une souris de chaque sexe mutante ou sauvage).
					}
			\label{fig:NisslResultat}
		\end{figure}
		\FloatBarrier

	\subsection{Distribution des neurones dans le cerveau}
	\label{ssec:neun}
		Pour visualiser l'organisation des neurones dans l'hippocampe des souris mutantes, j'ai révélé les neurones par \acrshort{ihc} en utilisant des anticorps dirigés contre \acrshort{neun}. \Acrshort{neun} (connu aussi sous le nom de Fox-3) est un marqueur du  corps cellulaire et des noyaux de presque tous les neurones, couramment utilisé dans les études neurologiques \cite{Guselnikova2015, Kim2009}. 
		
		Grâce à des mesures d'épaisseur de la couche pyramidale (CA1, CA3) et granulaire (Gyrus Denté) réalisées dans différentes régions de l'hippocampe sur des coupes rostrale, médiane et caudale, il apparaît que la structure de l'hippocampe est modifiée chez les souris mutantes par rapport aux souris sauvages (\cref{fig:NeuNResultat}). En comparant les mesures des coupes rostrales, on observe une différence dans la région du Gyrus Denté (72.50$\pm$18.17µm (\acrshort{wt}) vs 57.30$\pm$13.84µm (mutant), p-value : 0.0138) (\cref{fig:NeunQuantifNasal}).  Au niveau de coupes médianes, aucune différences ne ressort des différentes mesures effectuées (\cref{fig:NeunQuantifMilieu}). Pour les coupes caudales, on peut observer une différences d'épaisseur de la couche entre les souris \gls{wt} et \mcrd dans la région CA3 (111.20$\pm$20.39µm (\acrshort{wt}) vs 78.39$\pm$39.22 (mutant), p-value : 0.0129), et les deux sous-région du Gyrus denté (DG1 : 50.29$\pm$9.655 (\acrshort{wt}) vs 63.44$\pm$17.37 (mutant), p-value : 0.0244 ; et DG2 : 61.29$\pm$10.24 (\acrshort{wt}) vs 51.78$\pm$12.46 (mutant), p-value : 0.0367)(\cref{fig:NeunQuantifPost}).
		
		\begin{figure}[h]
			\begin{subfigure}[h]{0.329\textwidth} %M2 WT NeuN 3.tiff
				\caption{}
				\label{fig:NeuNIllu1}
				\includegraphics[width=\textwidth]{./Images/Immuno/NeuN/M2WT_NeuN_10x_1.jpg}
			\end{subfigure}
			\begin{subfigure}[h]{0.329\textwidth} %M2 WT NeuN 5.tiff
				\caption{}
				\label{fig:NeuNIllu2}
				\includegraphics[width=\textwidth]{./Images/Immuno/NeuN/M2WT_NeuN_10x_2.jpg}
			\end{subfigure}
			\begin{subfigure}[h]{0.329\textwidth} %M2 WT NeuN 7.tiff
				\caption{}
				\label{fig:NeuNIllu3}
				\includegraphics[width=\textwidth]{./Images/Immuno/NeuN/M2WT_NeuN_10x_3.jpg}
			\end{subfigure}
			\begin{subfigure}[h]{0.329\textwidth}
				\caption{}
				\label{fig:hippIllu}
				\includegraphics[width=\textwidth]{./Images/HippSchema.jpg}
			\end{subfigure}
			\begin{subfigure}[h]{0.329\textwidth}
				\caption{}
				\label{fig:NeunQuantifNasal}
				\includegraphics[width=\textwidth]{./Images/Immuno/NeuN/Quantif_Nasal.jpg}
			\end{subfigure}
			\begin{subfigure}[h]{0.329\textwidth}
				\caption{}
				\label{fig:NeunQuantifMilieu}
				\includegraphics[width=\textwidth]{./Images/Immuno/NeuN/Quantif_Milieu.jpg}
			\end{subfigure}
			\begin{subfigure}[h]{0.329\textwidth}
				\caption{}
				\label{fig:NeunQuantifPost}
				\includegraphics[width=\textwidth]{./Images/Immuno/NeuN/Qantif_Post.jpg}
			\end{subfigure}
			\caption{La distribution des neurones dans l'hippocampe est modifié par la mutation \mcrd.}
			\descfig{%
					Marquage de \acrshort{neun} sur tranche de cerveau. Illustration des différentes profondeurs choisies pour les mesures.
					\subref{fig:NeuNIllu1} Exemple de coupe rostrale.
					\subref{fig:NeuNIllu2} Exemple de coupe médiane.
					\subref{fig:NeuNIllu3} Exemple de coupe caudale.
					\subref{fig:hippIllu} Schéma des régions CA1, CA3, et DG 1\&2 dans lesquelles sont effectuées les mesures. Issue de Li et Pleasure, 2013 \cite{Li2013a}.
					\subref{fig:NeunQuantifNasal} Quantification de l'épaisseur de la couche pyramidale de différentes régions de l'hippocampe sur des coupes rostrales. 
					\subref{fig:NeunQuantifMilieu} Quantification sur des coupes médianes.
					\subref{fig:NeunQuantifPost} Quantification sur des coupes caudales.
					Ca : Cornus Ammonis, DG : Gyrus Denté. 
					Barre d'échelle : 2mm. 
					Test statistique : Test t de Student non apparié. 
					* : p<0.05, ** : p<0.01, *** : p<0.001. 
					n = 2 (\acrshort{wt}) et n = 3 (mutants).
					}
			\label{fig:NeuNResultat}
		\end{figure}
	\FloatBarrier
	
\section{Localisation et identification des cellules neurales exprimant \acrshort{musk}}
	\label{ssec:musk}
	Après avoir observé l'organisation globale du cerveau de souris et la distribution des neurones dans l'hippocampe, j'ai cherché à identifier les régions du cerveau exprimant \gls{musk}. Pour cela, j'ai eu recours à une technique d'immunohistochimie. L'anticorps utilisé est un anticorps polyclonal de lapin, dirigé contre le domaine extracellulaire de \gls{musk} (information donnée par le fournisseur).
	
	Malgré le fait que le niveau d'expression de \gls{musk} soit considéré comme faible dans le tissu nerveux d'après des études antérieurs (par northen blot principalement), j'ai quand même pu observer un marquage de cette protéine dans diverses régions discrètes du cerveau : Hippocampe (couche radiaire et moléculaire principalement) (\cref{fig:locaMuSKca1}), Corps calleux (\cref{fig:locaMuSKcc}), Habenula médiale (\cref{fig:locaMuSKhb}), Fasciculus retroflexus (\cref{fig:locaMuSKfr}), Capsule interne, Noyau caudé, 3ème ventricule ventrale, Cervelet principalement (\cref{fig:ImmunoMusk}). 
	
	\todo{Ajouter image coupe F437 belle}
	\begin{figure}[h] %Figure Immuno MuSK Résultats
		\begin{subfigure}[h]{0.99\textwidth}
			\caption{}
			\label{fig:locaMusK}
			\includegraphics[width=\textwidth]{./Images/Immuno/Musk/loca_MuSK.jpg}
		\end{subfigure}
		\begin{subfigure}[h]{0.245\textwidth}
			\caption{}
			\label{fig:locaMuSKcc}
			\includegraphics[width=\textwidth]{./Images/Immuno/Musk/MuSK_cc_50um.jpg}
		\end{subfigure}
		\begin{subfigure}[h]{0.245\textwidth}
			\caption{}
			\label{fig:locaMuSKca1}
			\includegraphics[width=\textwidth]{./Images/Immuno/Musk/MuSK_ca1_50um.jpg}
		\end{subfigure}
		\begin{subfigure}[h]{0.245\textwidth}
			\caption{}
			\label{fig:locaMuSKdg}
			\includegraphics[width=\textwidth]{./Images/Immuno/Musk/MuSK_dg_50um.jpg}
		\end{subfigure}
		\begin{subfigure}[h]{0.245\textwidth}
			\caption{}
			\label{fig:locaMuSKhb}
			\includegraphics[width=\textwidth]{./Images/Immuno/Musk/MuSK_hb_50um.jpg}
		\end{subfigure}
		\begin{subfigure}[h]{0.245\textwidth}
			\caption{}
			\label{fig:locaMuSKfr}
			\includegraphics[width=\textwidth]{./Images/Immuno/Musk/MuSK_fr_50um.jpg}
		\end{subfigure}
		\begin{subfigure}[h]{0.395\textwidth}
			\caption{}
			\label{fig:locaMusKCtrl}
			\includegraphics[width=\textwidth]{./Images/Immuno/Musk/loca_MuSK_ctrl.jpg}
		\end{subfigure}
		\caption{Localisation de \gls{musk} dans le cerveau.}
		\descfig{%
			\subref{fig:locaMusK} Immunomarquage de \gls{musk} sur coupe de cerveau entière. Coupe de souris mutante, mais le marquage est semblable chez les souris sauvages. %
			\subref{fig:locaMuSKcc} Grossissement sur l'organisation du marquage au niveau du corps calleux. %
			\subref{fig:locaMuSKca1} Grossissement sur l'organisation du marquage au niveau de la région CA1. L'épaississement vert correspond à la présence de noyau de la couche pyramidale. %
			\subref{fig:locaMuSKdg} Grossissement sur l'organisation du marquage au niveau du Gyrus Denté. %
			\subref{fig:locaMuSKhb} Grossissement sur l'organisation du marquage au niveau de l'habenula médiale. %
			\subref{fig:locaMuSKfr} Grossissement sur l'organisation du marquage au niveau du fasciculus retroflexus. %
			\subref{fig:locaMusKCtrl} Coupe contrôle du marquage de \gls{musk} dans un cerveau de souris \mcrd. Aucun marquage n'apparaît. %
			 alv : alveus hippocampus, cc : corps calleux, cp : pédoncule cérébral, ec : capsule externe, fr : fasciculus retroflexus, hp : hippocampe, or : stratum oriens de l'hippocampe. Barre d'échelle : 2 mm (\subref{fig:locaMusK}, \subref{fig:locaMusKCtrl}), 50µm (\subref{fig:locaMuSKcc}- \subref{fig:locaMuSKfr}).
			 	}
		\label{fig:ImmunoMusk}
	\end{figure}
	
	Dans ces régions, \gls{musk} est présent dans des prolongements cellulaires. La structure des cellules marquées par l'anticorps anti-\gls{musk} étant réminiscente de celles des astrocytes, un co-marquage de \gls{musk} et \gls{gfap}, une protéine spécifique du cytosquelette des astrocytes a été réalisé.
	
	Pour réaliser le marquage de \gls{gfap}, deux anticorps différents ont été utilisés. Le premier, fourni par l'équipe de C. Agulhon, provient de chez Millipore-Merck (\cref{table:Ac}, réf. MAB360) et a permis de visualiser des marquages spécifique. Le second anticorps, provenant de chez Abcam (\cref{table:Ac}, réf. ab4648) et commandé suite à une indisponibilité du premier anticorps, n'a révélé aucune marquage à différentes concentrations testées ($1{:}100$, $1{:}50$). Le premier anticorps a été utilisé dans l'ensemble des expériences.
	
	Suite au co-marquage de \gls{musk} et \gls{gfap}, on peut observer une colocalisation des deux marqueurs, ce qui semblerait indiqué que \gls{musk} est exprimé par les astrocytes (\cref{fig:ColocMuSK,fig:ColocGFAP,fig:ColocMuSK&GFAP}). 
	
	F. Semprez, un autre membre de l'équipe, a montré qu'en plus du cerveau, \gls{musk} est également présent dans la substance blanche de la moëlle épinière, et colocalise avec \gls{gfap} (\cref{fig:MuSKME}).
	
	%Images Coloc. MuSK GFAP
	\begin{figure}[h]
		\begin{subfigure}[h]{0.329\textwidth}
			\caption{}
			\label{fig:ColocMuSK}
			\includegraphics[width=\textwidth]{./Images/Immuno/Musk/MuSK-GFAP/M439_Mut_MuSK.jpg}
		\end{subfigure}
		\begin{subfigure}[h]{0.329\textwidth}
			\caption{}
			\label{fig:ColocGFAP}
			\includegraphics[width=\textwidth]{./Images/Immuno/Musk/MuSK-GFAP/M439_Mut_GFAP.jpg}
		\end{subfigure}
		\begin{subfigure}[h]{0.329\textwidth}
			\caption{}
			\label{fig:ColocMuSK&GFAP}
			\includegraphics[width=\textwidth]{./Images/Immuno/Musk/MuSK-GFAP/M439_Mut_MuSK_GFAP.jpg}
		\end{subfigure}
		\begin{subfigure}[h]{0.329\textwidth}
			\caption{}
			\label{fig:ColocZoom}
			\includegraphics[width=\textwidth]{./Images/Immuno/Musk/MuSK-GFAP/zoom10um.jpg}
		\end{subfigure}
		\begin{subfigure}[h]{0.329\textwidth}
			\caption{}
			\label{fig:MuSKME}
			\includegraphics[width=\textwidth]{./Images/Immuno/Musk/Moelle_MuSK_GFAP_150um.jpg}
		\end{subfigure}
		\caption{\gls{musk} et \gls{gfap} colocalisebt dabs les mêmes cellules.}
		\descfig{%
			Marquage de \gls{musk} (vert) et de \gls{gfap} (rouge) et \acrshort{dapi} (bleu) au niveau de la région CA1 de l'hippocampe. Le marquage de \gls{musk} colocalise avec \gls{gfap} dans toutes les régions du cerveau observées.
			\subref{fig:ColocMuSK} Marquage de \gls{musk}.
			\subref{fig:ColocGFAP} Marquage de \gls{gfap}.
			\subref{fig:ColocMuSK&GFAP} Superposition de \subref{fig:ColocMuSK} et \subref{fig:ColocGFAP}.
			\subref{fig:ColocZoom} Zoom de la partie encadré de \subref{fig:ColocMuSK&GFAP}. On peut observer que le prolongement marqué à la fois par \gls{musk} et \gls{gfap} semblent se mettrent en place autour d'un noyau (flèche blanche).
			\subref{fig:MuSKME} Co-marquage de \gls{musk} et de \gls{gfap} au niveau de la moëlle épinière.
			Barre d'échelle : 150µm \subref{fig:MuSKME} ; 20µm \subref{fig:ColocMuSK}, \subref{fig:ColocGFAP}, \subref{fig:ColocMuSK&GFAP} ; 10µm \subref{fig:ColocZoom}.
				}	
		\label{fig:colocalisation}
	\end{figure}
%\FloatBarrier
	
	Cependant, le marquage observé de \gls{musk} ne ressemble pas à ce que l'on peut observer au niveau de la \gls{jnm}, où \gls{musk} est observé uniquement au niveau du domaine post-synaptique. L'observation d'un marquage continue dans des prolongements astrocytaires nous à amené à nous poser la question de la spécificité de l'anticorps anti-\gls{musk}. Comme il n'existe pas d'autres anticorps dirigés contre \gls{musk} utilisables en \gls{ihc}, j'ai dû essayer d'autres approches afin de tenter de confirmer la spécificité du marquage. Tout d'abord, un co-marquage \gls{musk}/\acrshort{gfap} à été réalisé sur des sections de cerveau d'embryon de souris \gls{musk} KO (stade E18.5), la mutation étant létale après la naissance pour cause de défaillance respiratoire (\cref{fig:MuskEmbryon}). Sur seulement une coupe d'embryon \gls{wt}, un léger marquage ressemblant à celui observé chez les adultes est présent (\cref{fig:MuskE5WT,fig:MuskE5Marquage}), et aucun marquage n'est présent chez les animaux KO (\cref{fig:MuskE1KO}). Cela ne suffit pas à confirmer la spécificité du marquage de \gls{musk}, les astrocytes n'aparaissant qu'après la naissance. Pour tenter alors de confirmer la présence de \gls{musk} dans le cerveau, une immunoprécipitation a été alors envisagée (voire partie \cref{sec:IPresultat}).
	
	%Images Embryon
	\begin{figure}[h]
		\begin{center}
			\begin{subfigure}[h]{0.329\textwidth}
				\caption{}
				\label{fig:MuskE5WT}
				\includegraphics[width=\textwidth]{./Images/Immuno/Musk/Embryon/E5WT_50um_500px_df.jpg} 
			\end{subfigure}
			\begin{subfigure}[h]{0.329\textwidth}
				\caption{}
				\label{fig:MuskE5Marquage}
				\includegraphics[width=\textwidth]{./Images/Immuno/Musk/Embryon/E5_WT_MuSK_500px_Zoom_10um.jpg}
			\end{subfigure}
			\begin{subfigure}[h]{0.329\textwidth}
				\caption{}
				\label{fig:MuskE1KO}
				\includegraphics[width=\textwidth]{./Images/Immuno/Musk/Embryon/E1KO_50um_500px_df.jpg}
			\end{subfigure}
		\end{center}
		\caption{Pas de marquage de \gls{musk} sur embryon E18.5 WT et KO.}
		\descfig{%
			Marquage de \gls{musk} et \gls{gfap} sur des embryons (stade E18.5) de souris \gls{musk} KO (n = 2) et WT (n = 1). Aucun marquage de \gls{gfap} n'est visible car les astrocytes n'apparaissent qu'après la naissance. %
			\subref{fig:MuskE5WT} Embryon WT. Sur une coupe, un marquage ressemblant à celui de \gls{musk} observé chez les adultes était présent (flèches blanches). %
			\subref{fig:MuskE5Marquage} Agrandissement du marquage de \gls{musk}/\gls{gfap} encadré en \subref{fig:MuskE5WT}. %
			\subref{fig:MuskE1KO} Embryon KO. Le marquage observé (triangles blanc) n'est pas spécifique. %
			Images représentatives issues de la région du fornix dorsal. Barre d'échelle : 50µm \subref{fig:MuskE5WT} et  \subref{fig:MuskE1KO}, 10µm \subref{fig:MuskE5Marquage}. 
				}
		\label{fig:MuskEmbryon}
	\end{figure}
\FloatBarrier

	Afin d'identifier plus précisément le ou les types de cellules exprimant \gls{musk}, un co-marquage entre le récepteur et différents marqueurs a été réalisé sur des cultures primaires d'hippocampes, cultures issues d'embryon au stade E16 puis cultivées 14 jours, fournies par D. Carrel. Les marqueurs cellulaires sont \gls{gfap} pour marquer les astrocytes et \gls{map2} pour visualiser les dendrites des neurones. J'ai également testé un co-marquage \gls{musk} et \gls{glt1}, marqueur membranaire astrocytaire d'un transporteur du glutamate, afin de voir si \gls{musk} était localisé à la membrane, mais ce marquage n'a pas fonctionné.
	
	Les co-marquages de cultures cellulaires ont permis de confirmer la localisation observée sur les coupes de cerveaux de \gls{musk} : le récepteur est présent dans les prolongements astrocytaires (\cref{fig:CultureMGgfapmusk}). On peut voir sur les cultures les corps cellulaires des astrocytes, qui sont marqués à la fois par \gls{musk} et par \gls{gfap}, chose qui n'était pas visible lors de l'immunomarquage des coupes de cerveau.
	
	On observe également sur les images des prolongements cellulaires marqués par \gls{musk} qui ne sont pas co-marqués avec la \gls{gfap}. \gls{map2} est un marqueur des dendrites des neurones. Lors du co-marquage avec \gls{musk}, on peut voir sur les cultures certains prolongements sont à la fois marqués par \gls{map2} et par \gls{musk} (\cref{fig:CultureMMmap2}). Les prolongements non marqués ont une morphologie semblable à ceux marqués par \gls{gfap} : il s'agit ainsi du marquage d'astrocytes. En plus du marquage de prolongements, \gls{map2} va aussi être localisé dans le cytoplasme, autour de noyaux (\cref{fig:CultureMMmap2musk}, flèches blanches) qui va aussi co-localisé avec \gls{musk}. 
	
	La marquage de cultures cellulaires montre ainsi que non seulement \gls{musk} va être exprimé par les astrocytes, dans les prolongement mais aussi dans le corps cellulaire, comme observé sur les coupes de cerveaux, mais que le récepteur est aussi présent dans les neurones. \Gls{musk} est observé dans les dendrites, mais également dans le cytoplasme des cellules, chose qui n'était pas visible précédemment.
	
	\todo{Mettre MuSK GFAP MAP2 sur tte les images}	
	%Images cultures cellulaire
	\clearpage	
	\begin{figure}
		\begin{subfigure}[h]{0.329\textwidth}
			\caption{}
			\label{fig:CultureMGmusk}
			\includegraphics[width=\textwidth]{./Images/Immuno/Musk/Cultures/MuSK_50um.jpg}
		\end{subfigure}
		\begin{subfigure}[h]{0.329\textwidth}
			\caption{}
			\label{fig:CultureMGgfap}
			\includegraphics[width=\textwidth]{./Images/Immuno/Musk/Cultures/GFAP_50um.jpg}
		\end{subfigure}
		\begin{subfigure}[h]{0.329\textwidth}
			\caption{}
			\label{fig:CultureMGgfapmusk}
			\includegraphics[width=\textwidth]{./Images/Immuno/Musk/Cultures/GFAP-MuSK_50um.jpg}
		\end{subfigure}
		\begin{subfigure}[h]{0.329\textwidth}
			\caption{}
			\label{fig:CultureMMmusk}
			\includegraphics[width=\textwidth]{./Images/Immuno/Musk/Cultures/MuSK_20um.jpg}
		\end{subfigure}
		\begin{subfigure}[h]{0.329\textwidth}
			\caption{}
			\label{fig:CultureMMmap2}
			\includegraphics[width=\textwidth]{./Images/Immuno/Musk/Cultures/MAP2_20um.jpg}
		\end{subfigure}
		\begin{subfigure}[h]{0.329\textwidth}
			\caption{}
			\label{fig:CultureMMmap2musk}
			\includegraphics[width=\textwidth]{./Images/Immuno/Musk/Cultures/MuSK-MAP2_20um.jpg}
		\end{subfigure}
	\caption{Co-marquage de \gls{musk} et \gls{gfap} (astrocytes) ou \gls{musk} et \gls{map2} (neurones) sur une culture d'hippocampe.}
	\descfig{%
					Co-marquage de \gls{musk} (vert), \acrshort{dapi} (bleu) et \gls{map2} (rouge, \subref{fig:CultureMMmusk}-\subref{fig:CultureMMmap2musk}) et \gls{gfap} (rouge, \subref{fig:CultureMGmusk}-\subref{fig:CultureMGgfapmusk}).
					\subref{fig:CultureMGmusk} Marquage de \gls{musk}
					\subref{fig:CultureMGgfap} Marquage de \gls{gfap}
					\subref{fig:CultureMGgfapmusk} Merge 
					\subref{fig:CultureMMmusk} Marquage de \gls{musk}
					\subref{fig:CultureMMmap2} Marquage de \gls{map2}
					\subref{fig:CultureMMmap2musk} Merge. Flèches blanches : marquage cytoplasmique de \gls{map2} et de \gls{musk}.
					Barre d'échelle : 50µm (\subref{fig:CultureMGmusk}-\subref{fig:CultureMGgfapmusk}) ; 20µm (\subref{fig:CultureMMmusk}-\subref{fig:CultureMMmap2musk}).
					}
	\label{fig:MuSKMAP2GFAP}
	\end{figure}
	\FloatBarrier

\section{Immunoprécipitation}
\label{sec:IPresultat}
	Afin de confirmer la détection de \gls{musk} dans diverse région du cerveau, une immunoprécipitation suivie d'un \gls{wb} a été réalisée sur trois structures : l'hippocampe, le cervelet et le cortex de trois souris C57Bl/6 sauvage. Un \gls{wb} de \gls{musk} a déjà été réalisée par une autre équipe à partir de diverses structures du cerveau \cite{Garcia-Osta2006}, mais à partir d'extrait total de protéines et non d'une \gls{ip}, et avec des anticorps primaires différents. Une bande correspondante à \gls{musk} était révélé, bien que faible. Durant son stage, B. Somon a tenté de réaliser un \gls{ip} suivi d'un \gls{wb}, mais sans obtenir de résultats probant. Ici, la principale modification apportée au protocole précédemment utilisée par cette étudiante est l'utilisation de tampon RIPA (adjonction de déoxycholate de sodium et de \acrshort{sds} dans le tampon) qui permet une meilleure lyse et une meilleure préservation des protéines lors de l'extraction.
	
	Afin de vérifier si la technique fonctionnait, j'ai tout d'abord prélevé l'hippocampe, le cervelet et le cortex de trois souris C57Bl/6 puis réalisé une \gls{ip} de \gls{musk} à partir de ces tissus, qui a ensuite été migré dans un gel pour un \gls{wb}. En première expérience, comme l'anticorps utilisé pour l'\gls{ip} et la révélation était le même, un anticorps secondaire ciblant uniquement les chaines légères a été utilisé, afin d'éviter d'avoir trop de marquage non spécifique. Cependant, rien ne fut révélé, même le témoin positif étant très faible malgré un temps d'exposition important (supérieur à 30 minutes) (\cref{fig:WB-anti-LC}). Pour une deuxième expérience, la membrane a alors été deshybridée puis ré-incubée avec de l'anti-\gls{musk} ainsi qu'un autre anticorps secondaire dirigé contre les chaînes légères et lourdes d'\Glspl{ig} de lapin, qui à pour défaut d'être plus bruité (bande à 75kDa correspondant aux anticorps non couplés lors de l'\gls{ip}), mais qui permet de mieux révéler les bandes faiblement exprimées. De façon surprenante, des bandes de taille attendue (110kDa) ont été révélées dans l'hippocampe et le cervelet (\cref{fig:WBbon}, flèches noires), alors que dans le cortex une bande d'intensité plus faible semblait être présente (\cref{fig:WBbon}, tête de flèche noire), ce qui pourrait correspondre au fait qu'il n'y ait pas dans cette région du cerveau de marquage de \gls{musk} par \gls{ihc}. 
	
	Comme la taille des bandes ne correspondaient pas tout à fait à celle du témoin positif (extrait de cellules HEK293 transfectées avec un ADNc de \gls{musk}, à 130kDa, \cref{fig:WBbon}, flèche blanche), une troisième expérience a été réalisée avec des souris \mcrd et sauvages, issues de la même lignée. En effet, suite à la mutation, la protéine \gls{musk} à un poids moléculaire attendu de ~80kDa : si c'est bien \gls{musk} qui est détecté, on devrait alors avoir un décalage de poids entre les bandes. Cette fois-ci cependant, l'expérience ne s'est pas montrée concluante, rien n'a été révélé (\cref{fig:WB1erechec}). Enfin, afin de confirmer ce résultat, une quatrième et dernière expérience a finalement été tentée, avec comme témoin positif les extraits protéiques de la deuxième expérience (cervelet et cortex, plus assez d'extraits d'hippocampe pour recommencer). Aucun signal n'a été révélé, ni dans la troisième, ni dans la quatrième expérience(\cref{fig:WB1erechec}, \cref{fig:WBpasbon}).
	
	%Western Blot	
	\begin{figure}[h]
		\begin{center}
			\begin{subfigure}[h]{0.49\textwidth}
				\caption{}
				\label{fig:WB-anti-LC}
				\includegraphics[width=\textwidth]{./Images/WB/@LC_MuSK_30'.jpg} %Gel anti light chain
			\end{subfigure}
			\begin{subfigure}[h]{0.49\textwidth}
				\caption{}
				\label{fig:WBbon}
				\includegraphics[width=\textwidth]{./Images/WB/2018-04-09.jpg} %Gel Bon :)
			\end{subfigure}
			\begin{subfigure}[h]{0.49\textwidth}
				\caption{}
				\label{fig:WB1erechec}
				\includegraphics[width=\textwidth]{./Images/WB/2018-04-19.jpg} %Gel pas bon :(
			\end{subfigure}
			\begin{subfigure}[h]{0.49\textwidth}
				\caption{}
				\label{fig:WBpasbon}
				\includegraphics[width=\textwidth]{./Images/WB/2018-05-03.jpg} %Gel pas bon :(
			\end{subfigure}
		\end{center}
		\caption{L'\gls{ip} de \gls{musk} ne permet pas de confirmer la présence de \gls{musk} dans différentes structures du cerveau.}
		\descfig{%
			\subref{fig:WB-anti-LC} \gls{ip} suivie suivie de \gls{wb} sur 3 souris C57BL/6 révélée avec un anticorps dirigée contre les chaînes légères d'\gls{ig} de lapin. Seule la bande témoin au alentours de 130kDa est révélée. Exposition : 30 minutes. %
			\subref{fig:WBbon} \gls{ip} suivie de \gls{wb} sur 3 souris C57BL/6. Des bandes sont révélés aux alentours de 110kDa pour les régions de l'hippocampe et du cervelet (flèches noires). Une bande de faible intensité semble être présente pour le cortex (tête de flèche noire). Témoin positif : 15µL d'extrait cellulaire de cellules HEK293 transfectées, bande à 130kDa (flèche blanche). Exposition : 3 minutes. %
			\subref{fig:WB1erechec} \gls{ip} suivie de \gls{wb} sur 3 souris \mcrd (Mut) et 3 souris sauvages (WT) issues de la même souche. Exposition : 3 minutes.
			\subref{fig:WBpasbon} \gls{ip} suivie de \gls{wb} sur 3 souris \mcrd (Mut) et 3 souris sauvages (WT) issues de la même souche. Aucune bande n'est révélée, avec ou sans anticorps durant \gls{ip}. Aucune bande n'est révélé non plus chez le témoin positif : Extrait de cervelet et cortex issus de la première expérience. Hipp : Hippocampe, Ct : Cervelet, Cx : Cortex, + : Témoin positif, Ac- : Sans anticorps lors de l'\gls{ip}, Mut : Mutant, \acrshort{wt} : Sauvage. Poids moléculaire de \gls{musk} attendu : 110kDa. Exposition : 3 minutes. %
				}
		\label{fig:WBResultat}
	\end{figure}
\FloatBarrier

\section{Expression de \acrshort{musk} et \mcrd dans le cerveau}
\label{sec:ExpressionMuSK}
	Afin de tester l'impact potentiel de la delétion du \gls{crd} sur l'expression de la protéine, j'ai réalisé une \gls{qpcr} sur trois structures différentes : le cervelet, l'hippocampe gauche et droit. En effet, des résultats très préliminaires dans le laboratoire chez des 2 souris sauvages de souche C57Bl/6 (1 mâle et 1 femelle) avaient montrés une légère différence d'expression selon l'hémisphère du cerveau étudié. De plus, on retrouvait également une différence d'expression de \gls{musk} entre des individus de sexe opposé.Mon objectif a donc été de confirmer les différences d'expressions liées au genre et à l'hémisphère chez les souris \gls{wt} et de tester ces mêmes paramètres chez les souris \mcrd.
	
	Par manque d'individus au moment de l'expérience, je n'ai pas pu tester les conditions mâles \emph{versus} femelles. Cependant, on peut voir que l'expression de \gls{musk} varie entre les individus \gls{wt} et mutants (\cref{fig:TableAmpli}), avec l'expression chez les individus sauvages étant 100-1000 fois plus élevé que chez les souris \mcrd, dans les trois structures observées (\cref{fig:qPCRCompaWTMut}). L'expression du gène de ménage (\gls{26s}) est considéré comme identique entre les individus.
	
	Concernant la différences d'expression entre les différentes structures du cerveau, j'ai pu constaté que chez les individus \gls{wt} (\cref{fig:qPCRCompaWT}, \gls{musk} était plus exprimé dans l'hippocampe gauche que dans l'hippocampe droit (p-value : 0.0453) ou dans le cervelet (p-value : 0.0462). La différence entre l'hippocampe droit et le cervelet n'est pas significative (p-value : 0.0648).
	
	Chez les souris \mcrd, il n'y avait pas de différence significative dans l'expression de \gls{musk} dans les différentes structures (\cref{fig:qPCRCompaMut}), mais cette expression est cent fois plus faible que chez les souris sauvages. Ce point sera discuté plus tard dans la discussion.
	%\todo{lien discussion-résultats}
	
	\begin{figure}[h]
		\begin{center}
			\begin{subfigure}[h]{0.329\textwidth}
				\caption{}
				\label{fig:qPCRCompaWTMut}
				\includegraphics[width=\textwidth]{./Images/qPCR/Comp_WT-Mut.jpg}
			\end{subfigure}
			\begin{subfigure}[h]{0.329\textwidth}
				\caption{}
				\label{fig:qPCRCompaWT}
				\includegraphics[width=\textwidth]{./Images/qPCR/Comp_Struct_WT.jpg}
			\end{subfigure}
			\begin{subfigure}[h]{0.329\textwidth}
				\caption{}
				\label{fig:qPCRCompaMut}
				\includegraphics[width=\textwidth]{./Images/qPCR/Comp_Struct_Mut.jpg}
			\end{subfigure}
		\end{center}
		\caption{L'expression de \gls{musk} est fortement diminué chez les mutants.}
		\descfig{%
				\subref{fig:qPCRCompaWTMut} Comparaison de l'expression des différentes structures entre individus sauvages et mutants.
				\subref{fig:qPCRCompaWT} Comparaison de l'expression de \gls{musk} dans différentes structures chez les individus sauvages.
				\subref{fig:qPCRCompaMut} Comparaison de l'expression de \gls{musk} dans différentes structures chez les individus mutants.
				HG : Hippocampe Gauche, HD : Hippocampe Droit, Ct : Cervelet. n = 3 (\gls{wt}) et n = 4 (mutants). Test statistique : Test t de Student non apparié (\subref{fig:qPCRCompaWTMut}), et Test t de Student apparié (\subref{fig:qPCRCompaWT} \& \subref{fig:qPCRCompaMut}). 
				* : p<0.05, ** : p<0.01, *** : p<0.001. 
				}
		\label{fig:ExpressionMuSK}
	\end{figure}
\FloatBarrier

\section{Comportement}
\label{sec:Comportement}
Concernant les blessures préalablement observées sur les souris \mcrd mâles lors du stage de B. SOMON, elles n'ont pas pu être reproduites durant mon stage, alors que les animaux provenaient de 2 animaleries différentes, celle de la plateforme locale du site des Saints Pères et celle de l’ICM (Institut du Cerveau et de la Moëlle, Paris). Les animaux (mutants comme hétérozygotes) ont cependant été qualifiés de "plus sensible à leurs environnement que la normale" par les animaliers en charges de ceux-ci. On peut faire l'hypothèse que durant le stage de B. Somon, les souris, qui étaient à l'époque élevée dans une ancienne animalerie, étaient soumises à un stress plus important. Ce stress pourrait expliquer un comportement d'automutilation/hypergrooming chez les souris mutantes. 
%\todo{Resultats Laure strochlic + maze test}
