	Durant mon stage, j'ai pu étudier l'effets de la délétion du \gls{crd} de \gls{musk} sur l'organisation du cerveau, ainsi que la localisation du récepteur dans le tissu nerveux.
	
	Tout d'abord, au niveau de l'organisation générale du cerveau, j'ai montré grâce à une coloration de Nissl que celle-ci n'était pas impactée par la mutation du récepteur. Ensuite, grâce au marquage de \gls{neun}, j'ai pu montrer que l'hippocampe chez l'individu mutant allait être désorganisé, passant par une réduction de la taille de CA3 et une modification de l'organisation du Gyrus Denté.
	
	J'ai ensuite montré que \gls{musk} allait être exprimé dans des zones discrètes du cerveau (Hippocampe, Habenula, Fasciculus Retroflexus, Corps Calleux, Pédoncule cérébraux notamment) et que ce marquage prenait la forme de prolongement astrocytaire. En effet, le marquage de \gls{musk} et de \gls{gfap}, un marqueur des astrocytes, co-localisait. En regardant des cultures d'hippocampes marquées avec \gls{musk} et un marqueur cellulaire (\gls{gfap} ou \gls{map2}), j'ai pu montré que le récepteur allait être présent à la fois dans les prolongement astrocytaire et dans les dendrites des neurones.
	
	Enfin, j'ai pu montré que l'expression de \mcrd était perturbée par rapport à l'expression du récepteur sauvage, et j'ai également pu montré que l'expression de \gls{musk} différait entre les hippocampes droit et gauche d'un individu.
	
	Le rôle de \gls{musk} dans le cerveau reste pour l'instant très mal compris. 
	
	A finir.
	
	