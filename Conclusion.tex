Concernant l'organisation de l'hippocampe, je ne me suis intéressé qu'a la répartition des neurones dans l'hippocampe, et non à la densité cellulaires. Il pourrait être intéressant de comparer celle-ci entre des individus sauvages et mutants. De plus, le Gyrus Denté étant l'un des lieu de neurogenèse chez la souris adulte, et les \gls{wnt} étant impliquées dans la mémoire, il serait intéressant de faire une étude sur cette neurogenèse au travers un marquage au \gls{brdu}, afin de voir si celle-ci est perturbée chez la souris adulte mutante.

Un marquage de \gls{musk} par un autre anticorps pourrait permettre de lever le doute sur la spécificité de ce que j'ai pu observé. J'ai appris récemment que l'équipe de Lin Mei (Georgie, USA) possédait un anti-\gls{musk} utilisable en immunofluorescence. Je ne connais cependant pas quelle partie du récepteur cet anticorps reconnait.

Pour poursuivre l'études du rôle de \gls{musk} et des conséquences de la mutation \mcrd, l'étape suivante serait de réaliser des tests comportementaux, en collaboration avec une plateforme spécialisée de l'ICM. Ces tests permettraient d'observer les effets sur la mémoire, l'orientation spatiale, et le stress, de la mutation de \gls{musk}.

Sun \emph{et al.} avaient décrit la présence de \gls{musk} dans les astrocytes et les neurones par \gls{qpcr}. Les cellules provenaient d'embryon au stade E18. Ici, j'ai étudié l'expression de \gls{musk} chez des souris agées de 30 jours. Il serait intéressant d'observer l'évolution de l'expression de \gls{musk} au cours du développement, au moyen de \gls{qpcr} réalisés à différents stades embryonnaire et post-nataux. De plus, il faudrait produire des primers de \gls{qpcr} qui ne s'hybride pas dans la région du \gls{crd}. 