La spécificité du marquage reste un point crucial à confirmer. Plusieurs stratégies pourraient être mises en place pour cela. Tout d'abord, un marquage de \gls{musk} par un autre anticorps pourrait permettre de lever le doute sur la spécificité du marquage que j'ai pu observer. J'ai appris récemment que l'équipe de Lin Mei (Ohio, USA) possédait un anti-\gls{musk} utilisable en immunofluorescence \cite{Amenta2012}. Cet anticorps reconnait le domaine C-terminal du récepteur, situé du coté cytoplasmique. Une autre approche serait d'utiliser une stratégie basée sur un KO conditionnel de \gls{musk}. Hessert et al. en 2006 ont croisée une lignée de souris MuSK/LoxP avec une souris où le gène de la recombinase Cre est sous promoteur de la créatine kinase \cite{Hesser2006}. Dans ces souris, \gls{musk} est inactivé dans les muscles après la naissance, à la différences de KO constitutifs. En croisant la lignée \gls{musk}/LoxP avec une lignée exprimant Cre sous un promoteur neural (spécifique ou non des lignées astrocytaire ou neuronales), cela permettrait également d'étudier l'effet de la délétion de \gls{musk} dans le cerveau.

Concernant l'organisation de l'hippocampe, je ne me suis intéressé qu'à la répartition des neurones dans l'hippocampe, et non à la densité cellulaire, du fait du marquage trop dense. Il est donc utile de trouver une méthode de comptage des cellules, afin de les comparer entre individus sauvages et mutants. De plus, le Gyrus Denté étant l'un des lieux de neurogenèse chez la souris adulte, les \Glspl{wnt} étant impliquées dans ce phénomène, il serait intéressant de faire une étude de cette neurogenèse par un marquage au \gls{brdu}, afin de voir si elle est perturbée chez la souris adulte mutante.

A la \gls{jnm}, \gls{musk} est associé au co-récepteur \gls{lrp}4. Après visualisation de l'expression de la \textbeta{}-galactosidase sous contrôle du promoteur \gls{lrp}4, ce dernier ne semble être présent que dans la région de l'hippocampe \cite{Sun2016}. Plus précisément, \gls{lrp}4 est exprimé que dans les astrocytes de la couche moléculaire et du stratum lacunosum moleculare. Il serait envisageable de voir si \gls{musk} et \gls{lrp}4 sont co-localisées. De plus, l'agrine est une molécule essentielle pour la synaptogénèse du cerveau \cite{Cohen1997, Bose2000, Sun2016}. On peut donc se demander si la signalisation de \gls{musk} dans le cerveau est dépendante de ce ligand, ou si elle est stimulée par d'autres voies non identifiées.

Sun \emph{et al.} en 2016 avaient décrit la présence de \gls{musk} dans des astrocytes et des neurones par \gls{qpcr} \cite{Sun2016}, les cellules provenant d'embryons au stade E18. Ici, j'ai étudié l'expression de \gls{musk} chez des souris âgées de 30 jours. Il faudrait observer l'évolution de l'expression de \gls{musk} au cours du développement, au moyen de \gls{qpcr} réalisées à différents stades embryonnaires et post-nataux.

Pour poursuivre l'étude du rôle de \gls{musk} et des conséquences de la mutation \mcrd, l'étape suivante serait de réaliser des tests comportementaux, en collaboration avec une plateforme spécialisée de l'ICM. Des tests de labyrinthes surélevés en croix ont préalablement été réalisés par Laure Strochlic afin de tester l'anxiété des souris \mcrd et de voir les effets d'un traitement au \gls{licl} sur le comportement. Ces tests ont permis de montrer de façon préliminaire que les souris mutantes sont plus anxieuses que les souris sauvages, en plus de leur déficit de mémoire intermédiaire, et qu'il n'y a pas d'effets du \gls{licl} sur l'anxiété due à la mutation \mcrd. Les tests suivant à réaliser pourraient permettre d'observer les effets sur la mémoire, l'orientation spatiale, et le stress de la mutation de \gls{musk}. L'objectif \emph{in fine} de ce type d'étude est d'explorer les bases cellulaires du comportement lié à \gls{musk}. Il est à noter que ce serait la première étude qui attribuerait un rôle comportementale à \gls{musk}.