La spécificité du marquage reste un point crucial à confirmer. Plusieurs stratégies pourraient être mises en place afin de confirmer cela. Tout d'abord, un marquage de \gls{musk} par un autre anticorps pourrait permettre de lever le doute sur la spécificité de ce que j'ai pu observé. J'ai appris récemment que l'équipe de Lin Mei (Ohio, USA) possédait un anti-\gls{musk} utilisable en immunofluorescence \cite{Amenta2012}. Cet anticorps reconnait le domaine N-terminale du récepteur, situé du coté cytoplasmique. Ensuite, une stratégie basée sur un KO conditionnel de \gls{musk} pourrait être utilisée. Hessert et al. en 2006 ont généré une lignée de souris MuSK/LoxP croisée avec une souris où le gène de la recombinase Cre est sous promoteur de la créatine kinase \cite{Hesser2006}. Dans ces souris, \gls{musk} est inactivé dans les muscles après la naissance, à la différences de KO constitutifs. En croisant la lignée \gls{musk}/LoxP avec une lignée exprimant Cre sous un promoteur neural (spécifique ou non des lignées astrocytaire ou neuronales). Cette lignée permettrait également d'étudier l'effet de la délétion de \gls{musk} dans le cerveau.

Concernant l'organisation de l'hippocampe, je ne me suis intéressé qu'à la répartition des neurones dans l'hippocampe, et non à la densité cellulaire. Il pourrait être intéressant de comparer celle-ci entre des individus sauvages et mutants. De plus, le Gyrus Denté étant l'un des lieux de neurogenèse chez la souris adulte, et les \gls{wnt} étant impliquées dans la mémoire, il serait intéressant de faire une étude de cette neurogenèse par un marquage au \gls{brdu}, afin de voir si celle-ci est perturbée chez la souris adulte mutante.

A la \gls{jnm}, \gls{musk} est associé au co-récépteur \gls{lrp}4. Ce dernier ne semble être présent que dans la région de l'hippocampe, après visualisation de l'expression de la \textbeta{}-galactosidase sous contrôle du promoteur \gls{lrp}4. Plus précisément, \gls{lrp}4 ne semble être exprimé qu dans les astrocytes de la couche moléculaire et du stratum lacunosum moleculare \cite{Sun2016}. Il pourrait donc être intéressant de voir si les deux protéines sont colocalisées. L'agrine semble être essentielle pour la synaptogénèse du cerveau \cite{Cohen1997, Bose2000, Sun2016}. Il serait intéressant de voir si la signalisation de \gls{musk} est dépendante de ce ligand, ou si elle est stimulée par d'autres voies non identifiées.

Sun \emph{et al.} en 2016 avaient décrit la présence de \gls{musk} dans les astrocytes et les neurones par \gls{qpcr} \cite{Sun2016}. Les cellules provenaient d'embryons au stade E18. Ici, j'ai étudié l'expression de \gls{musk} chez des souris âgées de 30 jours. Il serait intéressant d'observer l'évolution de l'expression de \gls{musk} au cours du développement, au moyen de \gls{qpcr} réalisées à différents stades embryonnaires et post-nataux. De plus, il faudrait produire des primers de \gls{qpcr} qui ne s'hybrident pas dans la région du \gls{crd}.

Pour poursuivre l'étude du rôle de \gls{musk} et des conséquences de la mutation \mcrd, l'étape suivante serait de réaliser des tests comportementaux, en collaboration avec une plateforme spécialisée de l'ICM. Des tests de labyrinthe surélevé en croix avaient été réalisés par Laure Strochlic  afin de tester l'anxiété des souris \mcrd et voir les effets du traitement au \gls{licl}. Ces tests avaient permis de montrer de façon préliminaire que les souris mutantes étaient plus anxieuses que les souris sauvages en plus de leur déficit de mémoire intermédiaire, et qu'il n'y avait pas d'effets du \gls{licl} sur les effets de la mutation dans le cerveau. Les tests suivant à réaliser pourraient permettre d'observer les effets sur la mémoire, l'orientation spatiale, et le stress de la mutation de \gls{musk}. L'objectif \emph{in fine} de ce type d'étude est d'explorer les bases cellulaires du comportement lié à \gls{musk}. Il est à noter que ce serait la première étude qui attribuerait un rôle comportementale à \gls{musk}.